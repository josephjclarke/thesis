\chapter{Abstract}

This is a thesis about tipping points and early warning signals.
The tipping points investigated are related to various components of the climate-carbon system.
In contrast, the work on early warning signals has more generic applications, however in this thesis they are analysed in the context of the climate-carbon system.
The thesis begins with an introduction to the climate-carbon system as well as a discussion of tipping points in the
Earth system. Then a more mathematical summary of tipping points and early warning signals is given. An investigation into
the `compost bomb' is undertaken, in which the spatial structure of soils is accounted for. It is found that a hot summer could cause a compost bomb.
The effect of biogeochemical heating on the stability of the global carbon cycle is investigated and it is found to play only a small role.
The potential for instabilities in the climate-carbon cycle is further investigated when the dynamic behaviour of the ocean carbon cycle is accounted for. It is found
that some CMIP6 models may be close to having an unstable carbon cycle. Spatial early warning signals are investigated in the context of more rapidly forced
systems. It is found that spatial early warning signals perform better when the system is rapidly forced compared with time series based early warning signals.
The typical assumptions about white noise made when using early warning signals are also studied. It is found that time correlated noise may mask the
early warning signal. It is shown that a spectral analysis can avoid this problem.

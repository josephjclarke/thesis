\chapter{A Conceptual Model of the Carbon Cycle}
\label{chapter:conceptual_carbon_cycle}
\graphicspath{{box_ocean/figs/}}

In \cref{chapter:global_bomb}, I showed that there were conditions under which the climate was unstable,
driven by an instability in the terrestrial carbon cycle. In that chapter I found that the effects of
the biogeochemical feedback were small at the global scale. I also used a parameter $\chi_0$ to model the ocean,
which is a obvious simplification.

In this chapter then, I will construct a model of the climate-carbon system which is analytically tractable
with a dynamical ocean component but neglects the role of biochemical heating.

\section{Ocean Model}
I assume the ocean is made up of $N$ non-intereacting boxes.
The $i$th box absorbs atmospheric carbon over a timescale $\tau_i$ and recieves a fraction
$f_i$ of the atmospheric-ocean \ce{CO2} flux. I assume this flux is proportional to the difference in \ce{CO2}
from the atmospheric equilibrium value.

This leads to an equation of the form
\begin{equation}
  \label{eq:ocean_box_i}
  \dv{C_i}{t} = f_ik \Delta C_a(t) - \frac{C_i}{\tau_i},
\end{equation}
where $C_i$ is the change in ocean carbon stored in box $i$ and $k = \SI{0.2}{\per\year}$ gives the timescale
of the carbon flux.

For a given $C_a(t)$ \cref{eq:ocean_box_i} can be solved in quadratures to give
\begin{equation}
  \label{eq:solution_for_box_i}
  C_i(t) = \int_0^t f_ik e^{-s/\tau_i} \Delta C_a(t - u) \dd{u}.
\end{equation}
The overall ocean response is therefore
\begin{equation}
  \label{eq:ocean_response}
  \Delta C_o(t) = \sum_{i=1}^N \int_0^t f_ik e^{-s/\tau_i} \Delta C_a(t - u) \dd{u}.
\end{equation}

\Cref{eq:ocean_response} can then be fitted to observed changes in ocean carbon\todo{give details} to estimate the
parameters $f_i$ and $\tau_i$. Doing this for the two box case gives $f_1 = 0.9$, $\tau_1 = \SI{0.5}{\year}$ and
$\tau_2 = \SI{124}{\year}$. As a fraction, it must be the case that $f_2 = 1 - f_1$. For the one box case,
$\tau_1 = \SI{3.7}{\year}$.

\missingfigure{Plot of ocean carbon uptake with fits}.

\section{One Box Ocean}
Assume only one ocean box which responds linearly to elevated atmospheric \ce{CO2}. Set the
total carbon in the atmosphere-land-ocean system to $\mathcal{C}$.

Work with the system:
\begin{subequations}
  \label{eq:one_box_ocean}
  \begin{align}
    \dv{C_s}{t} &= \Pi(C_a) - r_0 C_s \left(\frac{C_a}{C_{a0}}\right)^{\mu} \\
    \dv{C_o}{t} &= k(C_a - C_{a0}) - \frac{C_o}{\tau} \\
    C_a &= \mathcal{C} - C_s - C_o
\end{align}
\end{subequations}
This leads to the Jacobian:
\begin{equation}
  \label{eq:jacobian_of_one_box}
    \bm{J} = 
    \begin{pmatrix}
    r_0 \left( \mu \frac{C_{s0}}{C_{a0}} - 1\right) - \Pi'(C_{a0}) & 
    \mu \frac{C_{s0}}{C_{a0}} - \Pi'(C_{a0}) \\
    -k & -k - \frac{1}{\tau}
    \end{pmatrix}
\end{equation}
The eigenvalues are given by finding the roots of the characteristic polynomial:
$\det(\bm{J} - \gamma \bm{I})$ = 0, where $\gamma$ is an eigenvalue. This equation is quadratic with 
two roots. Solving it leads to:
\begin{equation}
  \label{eq:eigenvalues_of_one_box_jac}
  \gamma_{\pm} = \frac{B \pm \sqrt{\Delta}}{2\tau C_{a0}}
\end{equation}
where
\begin{equation}
  \label{eq:B_in_one_box}
  B = -k \tau  C_{a0}-\Pi'(C_{a0}) \tau  C_{a0}-r_0 \tau  C_{a0}-C_{a0}+\mu  \Pi_0 \tau
\end{equation}
and
\begin{equation}
  \label{eq:discriminant_from_one_box}
  \Delta = \left(k \tau  C_{a0} +\Pi'\tau  C_{a0}+r_0 \tau  C_{a0}+C_{a0}-\mu  \Pi_0 \tau \right)^2-4 \left(k r_0 \tau ^2 C_{a0}^2+\Pi' \tau  C_{a0}^2-\mu  \Pi_0 \tau  C_{a0} +r_0 \tau  C_{a0}^2\right)
\end{equation}
is the determinant. I have used $\Pi_0 = r_0 C_{s0}$.

The behaviour of this system will depend on the sign of $\Delta$.

\subsection{Real Eigenvalues}
If $\tau < 1/r_0$ then $\Delta > 0$ for all values of $\mu$. $1/r_0 \approx 30$ years so this represents the fast ocean response.
To find the instability, we look for when $\gamma_+ > 0$. Under the assumption of a fast ocean response,
\begin{equation}
  \label{eq:instability_condition_one_box_fast}
  \mu^* = \frac{C_{a0}}{C_{s0}} + \frac{C_{a0}}{\Pi_0} \dv{\Pi}{C_a} + \frac{C_{a0}}{C_{s0}} k\tau.
\end{equation}
This is essentially the condition in \cref{eq:mu_infinity} if we make the identification $\chi_0 = k \tau$.
\subsection{Complex Eigenvalues}
If $\Delta < 0$, which is to say if the ocean response is considered on a long timescale, the stabilty of the system is given by the sign of $B$.
This leads to the condition
\begin{equation}
  \label{eq:instability_condition_one_box_slow}
  \mu^* =\frac{C_{a0}}{C_{s0}} + \frac{C_{a0}}{\Pi_0} \dv{\Pi}{C_a} + \frac{C_{a0}}{C_{s0}} k\tau\left(
     1 + \frac{1}{k\tau}
  \right) \frac{1}{r_0\tau}
\end{equation}
This corresponds to the eigenvalues crossing the imaginary axis and is thus a Hopf bifurcation.

\subsection{Comparison with Data}
For the one box case, it was found that $\tau r_0 \approx 0.1$ which means that the system is in the `real eigenvalues' case. The value of $\chi_0$ becomes
$0.7$.

\missingfigure{One Box Bifurcation Diagram}

\section{Two Box Model}
Suppose I modify \cref{eq:one_box_ocean} to include an extra ocean box. I imagine a fast box and a slow box. In principle, I could repeat the
analysis above, by finding the eigenvalues of a $3 \times 3$ matrix. This is possible in principle but involves solving a cubic polynomial which
is challenging analytically.

Instead, I will use the timescale seperation between the two ocean boxes to make some remarks about the qualitative behaviour of the system.
As the slow box is much slower than the fast box, over short timescales it can be ignored. So if $\mu$ were to be slowly increased through
the condition given in \cref{eq:instability_condition_one_box_fast} there would still be an instability.

However, on longer timescales, the system will still behave like a one box model but now the box will be the slow ocean component. As this is the slow box,
it will have $\Delta < 0$ and so there will be oscillatory behaviour. Hence I expect the bifurcation to occur at a similar place as in the one box fast ocean model,
but to have the character of a Hopf bifurcation.

To be more formal, consider the system
\begin{subequations}
  \begin{align}
    \label{eq:two_box_ocean}
    \dv{C_s}{t} &= \Pi(C_a) - r_0 C_s \left(\frac{C_a}{C_{a0}}\right)^{\mu} \\
    \dv{C_1}{t} &= fk(C_a - C_{a0}) - \frac{C_1}{\tau} \\
    \dv{C_2}{t} &= (1-f)k(C_a-C_{a0}) - \epsilon\frac{C_2}{\tau} \\
    C_a &= \mathcal{C} - C_s - C_1 - C_2
  \end{align}
\end{subequations}
where $f$ is the fraction of carbon taken up by the fast response and $\epsilon$ is the ratio of the timescale of the fast box to the slow box, so $\epsilon \ll 1$. If
$C_2$ can be taken as approximately constant, then there is an instability when $\mu^*_{\mathrm{fast}} = \frac{C_{a0}}{C_{s0}} + \frac{C_{a0}}{\Pi_0} \dv{\Pi}{C_a} + \frac{C_{a0}}{C_{s0}} kf\tau$.

Now working on the slower timescale where $C_1$ is approximately constant the eigenvalues of the Jacobian of $C_s$ and $C_2$ can be evaluated at $\mu = \mu^*_{\mathrm{fast}}$. In this case, the eigenvalues
have the form
\begin{equation}
  \label{eq:slow_eigenvalues}
  \gamma_{\pm} = \frac{f k r_0 \tau ^2+\sqrt{\left(-f k r_0 \tau ^2-f k \tau +k \tau +\epsilon \right)^2-4 \left(-f k r_0 \tau ^2-f k r_0 \tau ^2 \epsilon +k r_0 \tau ^2\right)}+f k \tau -k \tau -\epsilon }
  {2 \tau }.
\end{equation}
For oscaillatory solutions, the arguemnt of the square root must be negative, which requires
\begin{equation}
  \label{eq:epsilon_requirement}
  \epsilon < 2 \sqrt{-f^2 k^2 r_0 \tau ^3+f k^2 r_0 \tau ^3-f k r_0 \tau ^2+k r_0 \tau ^2}-f k r_0 \tau ^2+f k \tau -k \tau.
\end{equation}
For $f = 0.3$, $\tau = 5$ this implies that $\epsilon < 0.06$, which is consistent with the assumption that the second ocean box is slow. This value of $\epsilon$ implies a second box timescale
of about 80 years.
\subsection{Computation of Bifurcation Point}
There is another mode of attack however. The characteristic polynomial of the Jacobian of \cref{eq:two_box_ocean} will be of the form
\begin{equation}
  \label{eq:generic_cubic}
  \gamma^3 + a_1 \gamma^2 + a_2 \gamma + a_3 = 0.
\end{equation}
At the Hopf bifurcation, $\gamma$ is purely imaginary, so set $\gamma = i\lambda$. Then \cref{eq:generic_cubic} becomes
\begin{equation}
  \label{eq:cubic_with_imaginary_root}
  -i\lambda^3 - a_1 \lambda^2 + i a_2 \lambda + a_3 = 0. 
\end{equation}
Demanding that both real and imaginary parts of \cref{eq:cubic_with_imaginary_root} are zero, leads to $\lambda = \pm \sqrt{a_3/a_1}$ and $\lambda = \pm \sqrt{a_2}$. That these two expressions
are equal leads to the condition that $a_3 = a_1a_2$, which can then be solved for $\mu$. Doing this gives
\begin{equation}
  \label{eq:mu_two_box}
  \begin{split}
  \mu^* = &\frac{C_{a0}}{2 \Pi_0 \tau  (1+\epsilon)}
  \Biggl(
  -f k \tau  (1-\epsilon)+r_0 \tau  (2 + k \tau +2 \epsilon)+k \tau  (2+\epsilon)+(1+\epsilon) (1+2 \Pi' \tau +\epsilon)\\
  &\pm\sqrt{
    \begin{split}
    f^2 k^2 \tau ^2 (1-\epsilon)^2-&2 f k \tau  (1-\epsilon) \left(r_0 \tau  (k \tau +2 \epsilon +2)-k \tau  \epsilon -(1+\epsilon)^2\right)\\+
    \left(1+k r_0 \tau ^2-k \tau  \epsilon -\epsilon ^2\right)^2
  \end{split}}
  \Biggr)
  \end{split}
\end{equation}
or
\begin{equation}
  \label{eq:mu_two_box_zero_eps}
  \begin{split}
  \mu^* \sim &\frac{C_{a0}}{\Pi_0}\Biggl(
    \frac{1}{2\tau} + k(1 - \frac{1}{2}f + r_0 \tau) + r_0 + \Pi'\\
  &\pm\frac{1}{2}\sqrt{\frac{1}{\tau^2} + 2kf + k\tau(kf^2 + 2 (1 - 2f)r_0  - 2 k r_0\tau f  + k r_0^2 \tau^2)}\,\Biggr)
\end{split}
\end{equation}
as $\epsilon \rightarrow 0$.
There are 2 values of $\mu$ that satisfy the consistency condition --- there are two Hopfs? (Sub/supercritical?)
In the case where $\epsilon = 0$ and $f = 1$, this reduces to the one box model case. In this case \cref{eq:mu_two_box} becomes
\begin{equation}
  \label{eq:mu_zero_one}
  \mu^*_{\pm} = \frac{C_{a0} \left(\pm\left(-k r_0 \tau ^2+k \tau +1\right)+r_0 \tau  (k \tau +2)+k \tau +2 \Pi' \tau +1\right)}{2 \Pi_0 \tau}
\end{equation}
or
\begin{align}
  \label{eq:mu_zero_one_cases}
  \mu_+^* &= \frac{C_{a0}}{C_{s0}} + \frac{C_{a0}}{\Pi_0}\dv{\Pi}{C_a} + \frac{C_{a0}}{C_{s0}} \left(\frac{k}{r_0} + \frac{\tau^{-1}}{r_0}\right) \\
  \mu_-^* &= \frac{C_{a0}}{C_{s0}} + \frac{C_{a0}}{\Pi_0}\dv{\Pi}{C_a} + \frac{C_{a0}}{C_{s0}}k\tau 
\end{align}
which correspond to the conditions (\cref{eq:instability_condition_one_box_fast,eq:instability_condition_one_box_slow}) derived for the one box case.

\missingfigure{Eigenvalues of Jacobian Demonstrating Spurious Root}
\missingfigure{Bifurcation Diagram with fitted parameters}

\section{IMOGEN}
\subsection{IMOGEN description}
\subsection{Bifurcations in IMOGEN}
\missingfigure{Bifurcation Diagram for IMOGEN}
\subsection{Variability in IMOGEN}
\begin{subequations}
  \label{eq:climate_system}
  \begin{align}
    c_1 \dv{T_1}{t} &= \frac{Q_{2\times}}{\log 2} \log \frac{C_a}{C_{a0}} - \frac{Q_{2\times}}{ECS}T_1 -\gamma (T_1 - T_2) + \sigma_Q \xi(t)\\
    c_2 \dv{T_2}{t} &= -\gamma(T_1 - T_2) \\
    \dv{C_s}{t}     &= \Pi(C_a) - C_s r_0 e^{\alpha T_1} \\
    \dv{C_o}{t}     &= \mathrm{IMOGEN} \\
    C_a &= \mathcal{C} - C_s - C_o
  \end{align}
\end{subequations}
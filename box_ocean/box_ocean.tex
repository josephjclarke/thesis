\chapter{A Conceptual Model of the Carbon Cycle}
\label{chapter:conceptual_carbon_cycle}
\graphicspath{{box_ocean/figs/}}

In \cref{chapter:global_bomb}, I showed that there were conditions under which the climate was unstable,
driven by an instability in the terrestrial carbon cycle. In that chapter I found that the effects of
the biogeochemical feedback were small at the global scale. I also used a parameter $\chi_0$ to model the ocean,
which is a obvious simplification.

In this chapter then, I will construct a model of the climate-carbon system which is analytically tractable
with a dynamical ocean component but neglects the role of biochemical heating.

\section{Ocean Model}
I assume the ocean is made up of $N$ non-intereacting boxes.
The $i$th box absorbs atmospheric carbon over a timescale $\tau_i$ and recieves a fraction
$f_i$ of the atmospheric-ocean \ce{CO2} flux. I assume this flux is proportional to the difference in \ce{CO2}
from the atmospheric equilibrium value.

This leads to an equation of the form
\begin{equation}
  \label{eq:ocean_box_i}
  \dv{C_i}{t} = f_ik \Delta C_a(t) - \frac{C_i}{\tau_i},
\end{equation}
where $C_i$ is the change in ocean carbon stored in box $i$.

For a given $C_a(t)$ \cref{eq:ocean_box_i} can be solved in quadratures to give
\begin{equation}
  \label{eq:solution_for_box_i}
  C_i(t) = \int_0^t f_ik e^{-s/\tau_i} \Delta C_a(t - u) \dd{u}.
\end{equation}
The overall ocean response is therefore
\begin{equation}
  \label{eq:ocean_response}
  \Delta C_o(t) = \sum_{i=1}^N \int_0^t f_ik e^{-s/\tau_i} \Delta C_a(t - u) \dd{u}.
\end{equation}
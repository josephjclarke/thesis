\chapter{A Conceptual Model of the Carbon Cycle}
\label{chapter:conceptual_carbon_cycle}
\graphicspath{{box_ocean/figs/}}

In \cref{chapter:global_bomb}, I showed that there were conditions under which the climate was unstable,
driven by an instability in the terrestrial carbon cycle. In that chapter I found that the effects of
the biogeochemical feedback were small at the global scale. I also used a parameter $\chi_0$ to model the ocean,
which is a obvious simplification.

In this chapter then, I will construct a model of the climate-carbon system which is analytically tractable
with a dynamical ocean component but neglects the role of biochemical heating.

\section{Ocean Model}
I assume the ocean is made up of $N$ non-intereacting boxes.
The $i$th box absorbs atmospheric carbon over a timescale $\tau_i$ and recieves a fraction
$f_i$ of the atmospheric-ocean \ce{CO2} flux. I assume this flux is proportional to the difference in \ce{CO2}
from the atmospheric equilibrium value.

This leads to an equation of the form
\begin{equation}
  \label{eq:ocean_box_i}
  \dv{C_i}{t} = f_ik \Delta C_a(t) - \frac{C_i}{\tau_i},
\end{equation}
where $C_i$ is the change in ocean carbon stored in box $i$ and $k = \SI{0.2}{\per\year}$ gives the timescale
of the carbon flux.

For a given $C_a(t)$ \cref{eq:ocean_box_i} can be solved in quadratures to give
\begin{equation}
  \label{eq:solution_for_box_i}
  C_i(t) = \int_0^t f_ik e^{-s/\tau_i} \Delta C_a(t - u) \dd{u}.
\end{equation}
The overall ocean response is therefore
\begin{equation}
  \label{eq:ocean_response}
  \Delta C_o(t) = \sum_{i=1}^N \int_0^t f_ik e^{-s/\tau_i} \Delta C_a(t - u) \dd{u}.
\end{equation}

\Cref{eq:ocean_response} can then be fitted to observed changes in ocean carbon\todo{give details} to estimate the
parameters $f_i$ and $\tau_i$. Doing this for the two box case gives $f_1 = 0.9$, $\tau_1 = \SI{0.5}{\year}$ and
$\tau_2 = \SI{124}{\year}$. As a fraction, it must be the case that $f_2 = 1 - f_1$. For the one box case,
$\tau_1 = \SI{3.7}{\year}$.

\missingfigure{Plot of ocean carbon uptake with fits}.

\section{One Box Ocean}
Assume only one ocean box which responds linearly to elevated atmospheric \ce{CO2}. Set the
total carbon in the atmosphere-land-ocean system to $\mathcal{C}$.

Work with the system:
\begin{subequations}
  \label{eq:one_box_ocean}
  \begin{align}
    \dv{C_s}{t} &= \Pi(C_a) - r_0 C_s \left(\frac{C_a}{C_{a0}}\right)^{\mu} \\
    \dv{C_o}{t} &= k(C_a - C_{a0}) - \frac{C_o}{\tau} \\
    C_a &= \mathcal{C} - C_s - C_o
\end{align}
\end{subequations}
This leads to the Jacobian:
\begin{equation}
  \label{eq:jacobian_of_one_box}
    \bm{J} = 
    \begin{pmatrix}
    r_0 \left( \mu \frac{C_{s0}}{C_{a0}} - 1\right) - \Pi'(C_{a0}) & 
    \mu \frac{C_{s0}}{C_{a0}} - \Pi'(C_{a0}) \\
    -k & -k - \frac{1}{\tau}
    \end{pmatrix}
\end{equation}
The eigenvalues are given by finding the roots of the characteristic polynomial:
$\det(\bm{J} - \gamma \bm{I})$ = 0, where $\gamma$ is an eigenvalue. This equation is quadratic with 
two roots. Solving it leads to:
\begin{equation}
  \label{eq:eigenvalues_of_one_box_jac}
  \gamma_{\pm} = \frac{B \pm \sqrt{\Delta}}{2\tau C_{a0}}
\end{equation}
where
\begin{equation}
  \label{eq:B_in_one_box}
  B = -k \tau  C_{a0}-\Pi'(C_{a0}) \tau  C_{a0}-r_0 \tau  C_{a0}-C_{a0}+\mu  \Pi_0 \tau
\end{equation}
and
\begin{equation}
  \label{eq:discriminant_from_one_box}
  \Delta = \left(k \tau  C_{a0} +\Pi'\tau  C_{a0}+r_0 \tau  C_{a0}+C_{a0}-\mu  \Pi_0 \tau \right)^2-4 \left(k r_0 \tau ^2 C_{a0}^2+\Pi' \tau  C_{a0}^2-\mu  \Pi_0 \tau  C_{a0} +r_0 \tau  C_{a0}^2\right)
\end{equation}
is the determinant. I have used $\Pi_0 = r_0 C_{s0}$.

The behaviour of this system will depend on the sign of $\Delta$.

\subsection{Real Eigenvalues}
If $\tau < 1/r_0$ then $\Delta > 0$ for all values of $\mu$. $1/r_0 \approx 30$ years so this represents the fast ocean response.
To find the instability, we look for when $\gamma_+ > 0$. Under the assumption of a fast ocean response,
\begin{equation}
  \label{eq:instability_condition_one_box_fast}
  \mu^* = \frac{C_{a0}}{C_{s0}} + \frac{C_{a0}}{\Pi_0} \dv{\Pi}{C_a} + \frac{C_{a0}}{C_{s0}} k\tau.
\end{equation}
This is essentially the condition in \cref{eq:mu_infinity} if we make the identification $\chi_0 = k \tau$.
\subsection{Complex Eigenvalues}
If $\Delta < 0$, which is to say if the ocean response is considered on a long timescale, the stabilty of the system is given by the sign of $B$.
This leads to the condition
\begin{equation}
  \label{eq:instability_condition_one_box_slow}
  \mu^* =\frac{C_{a0}}{C_{s0}} + \frac{C_{a0}}{\Pi_0} \dv{\Pi}{C_a} + \frac{C_{a0}}{C_{s0}} k\tau\left(
     1 + \frac{1}{k\tau}
  \right) \frac{1}{r_0\tau}
\end{equation}
This corresponds to the eigenvalues crossing the imaginary axis and is thus a Hopf bifurcation.
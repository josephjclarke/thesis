\chapter{Introduction}
\graphicspath{{introduction/figs}}

\section{Tipping Points}
Some general remarks about tipping points.

\section{Tipping Points in the Earth System}
\subsection{Evidence of Tipping Points from Paleoclimate}
\subsection{Tipping Elements}

\section{A Typology of Tipping Points}
\subsection{Bifurcation}
\subsubsection{Theory}
If $\dot{x} = f(x)$, show that $f'(x_{tip}) = 0$. Normal forms
\subsubsection{Stommel's AMOC Model}
\subsection{Noise}
\subsubsection{Theory}
Derive Kramers rate?
\subsubsection{D.O. Events}
\subsection{Rate}
\subsubsection{Theory}
?
\subsubsection{Compost Bomb}
\subsection{Shock}
\subsubsection{Theory}
Calculate the distance to the boundary for a general fixed point. Suppose $x^*$ is a fixed point with basin $\mathcal{B}$. Let $d(a,b)$ be the distance from
$a$ to $b$ .Then $\Delta = \inf \{d(x,x^*) : x \in \partial\mathcal{B}\}$.
\subsubsection{An Example}
Consider the Allee Effect.
\begin{equation}
  \label{eq:allee_effect}
  \dot{x} = x\left(x-1\right)\left(1-\frac{x}{k}\right).
\end{equation}
Assume $k > 1$. This has equilibria at $x_1 = 0$, $x_2 = 1$ and $x_3 = k$. The equilibria $x_1,x_3$ are stable. If the system is
initially in equilibrium at $x_3$, the distance to the basin boundary is $\Delta = |x_3 - x_2| = k - 1$. Hence if the system recieves a kick of
this magnitude or greater then the system will transition to the extinction state.

\section{Early Warning Signals}
Systems behaving 1D. Critical Slowing Down. Use of variance and
autocorrelation. Other methods. Fluctuation dissipation theorem.
Resiliance. Critical Speeding up


\section{Non-autonomous Topics}
Overshoot. Timescales. Slow fast systems.
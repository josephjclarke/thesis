\chapter{Introduction}
\graphicspath{{introduction/figs}}

\section{Tipping Points}
Take a pan of water and heat it up. As its temperature rises, some of the water's properties may change. For example, the viscosity and heat flux to the
environment may change. However, the water still remains as water and these changes will vary continuously with the water temperature. However, if the water is heated beyond
a critical temperature of \SI{100}{\degreeCelsius} something more dramatic happens and the water boils away. The properties of the resulting steam are quite different
to the original water --- a \emph{qualitiative} change has occured.

It is this sort of qualitative change that this thesis shall be concerned with. Whilst the boiling of water is a phase transition \parencite{Goldenfeld1992}, I shall use the broader terminology
of \emph{tipping points} \parencite{Lenton2008} or \emph{critical transitions} \parencite{Rahmstorf1995} or \emph{abrupt changes} \parencite{Alley2003} to describe these phenomena. I shall largely stick
to the term tipping point, or just tipping, which has its origins in the idea that when an object is leant over it will return to its original position unless it is leant over sufficently far when the object
will tip over.

\subsection{Definitions}

Although it can be a vague term, the concept of a small nudge leading to a large change generally forms a part of most formal definitions of tipping points.
For example,~\cite{Lenton2008} defines the occurance of a tipping
point as when a control parameter, there is a critical valuem $\rho_{\mathrm{crit}}$ of a control parameter, $\rho$, above which any significant variation, $\delta \rho > 0$ leads to a qualitative change
$\hat{F}$ of a system feature $F$, after some observation time $T > 0$. They write this mathematically as
\begin{equation}
  \label{eq:lenton_tipping_definition}
  |F(\rho \geq \rho_{\mathrm{crit}} + \delta \rho | T) - F(\rho|T)| \geq \hat{F} > 0.
\end{equation}
More recently,~\cite{ArmstrongMcKay2022} updated this definition by requiring $\delta \rho$ to be within the natural variability of the system. They also require that changes become self perpetuating
beyond $\rho_{\mathrm{crit}}$ as a result of the asymmetry of relevant feedbacks and that the tipping leads to `substantial' impacts. The IPCC \parencite{AR6} defines tippings differently still
as `a level of change in system properties beyond which a system reorganises, often in a nonlinear manner, and does not return to the initial state even if the drivers of the change are abated'.
Others, such as~\cite{Wang2023}, inspired by~\cite{Kopp2016}, insist that tipping points should occur on fast time scales.

Each of these definitions are deficent in their own way. The IPCC definition allows for linear behaviour, although tipping is a fundamentally nonlinear phenomenon. It also insists on \emph{hysteresis},
which is the property that the system does not return to its initial state after abating the drivers, although not all examples of tipping --- including boiling water --- experience this.
The complex definition in~\cite{Lenton2008} views tipping in terms of changes in parameters of a system, and not something a system can do spontaneouslty, which excludes certain types of
tipping (see \cref{sec:tipping_typology}). The third definition, that of~\cite{ArmstrongMcKay2022}, suffers from the problems of~\cite{Lenton2008} and also requires `substantial' impacts, which
artifically restricts the abrupt shifts that can be investigated. Finally,the requirement of~\cite{Wang2023} that tipping points be realised quickly is problematic for Earth System investigations
as some of the tipping points of interest will be realised slowly.

In this thesis, I will use the term tipping point in the sense that a system undergoes a tipping point when it experiences a qualitative change in its properties. Although this is a vague definition,
it enables the term tipping point to include all dynamics of interest. I am adopting Potter Stewart's philosophy that although a tipping point might be hard to define,
`I know it when I see it' \parencite{Stewart1964}.

\subsection{Examples of Tipping}

These sorts of nonlinear changes have long facinated scientists. Physicists have investigated phase changes not just in terms of the boiling and freezing of substances but also
in magnetic materials \parencite{Ising1925,Onsager1944}, superconductivity \parencite{Landau1965} and percolation theory \parencite{Flory1941}. Each of these are different processes, but it is a
remarkable fact that near the phase transition different systems can be dynamically very similar, a phenomenon known as universality \parencite{Wilson1983}. This sort of idea --- that different examples
of tipping points across totally different systems can share many common features as been an important driving force in the theory of tipping points. 

Tipping points are also important in biology. In medicine, tipping point theory has been used to help understand asthma attacks \parencite{Donovan2022} and sleeping dynamics \parencite{Skeldon2014}.
Tipping points have proven very important in ecology. In an influential article, \parencite{May1976}, the ecologist Robert May noted that even very simple nonlinear models have rich dynamics and are capable
of experiencing tipping points (although he did use that term).~\cite{Holling1973} introduced the idea of resilience which is to do with the ability of a system to resist tipping. Tipping points have
been found in a range of ecosystems \parencite{Scheffer2001,Dakos2019}. The transition to turbid state in lakes \parencite{Scheffer1993} and the collapse of plant-polinator communities
\parencite{Lever2014} are both examples of ecological tipping points, however other studies \parencite{Hillebrand2020} have challenged this.

Notions of multiple equilibria and the ability to transition between them has been used to explain patterns seen in nature. This was introduced by Turing \parencite{Turing1952} to explain
patterns found in certain plants and animals. Since then, it has been used to explain patterns in ecosystems \parencite{Rietkerk2008}.

\section{Tipping Points in the Earth System}
Whilst part of this thesis is applicable to many systems, the focus of this work has been on applications to the Earth system. In this section I want to give some background on these tipping
points, including outlining some mechanisms. I will focus on a few key sub subsistems. I will then discuss some of the evidence for tipping behaviour in the Earth's past.

The most recent IPCC report \parencite{AR6} finds it `unequivocal that human influence has warmed the atmosphere, ocean and land'. Humans have done this through the release of greenhouse
gases, most importantly carbon dioxide (\ce{CO2}), and also through land use change. This has led to observable changes in the Earths climate. Most obvious is the changes in global mean surface
temperature, with a most likely temperature increase of \SI{1.07}{\kelvin} \parencite{AR6} in the global mean but with clear regional differences \parencite{Morice2021} such as
artic amplification as well as more warming over land than over ocean. Other effects include rise of around \SI{0.2}{\meter} in sea levels \parencite{Frederikse2020} and increase in heavy precipitation
events \parencite{Fischer2016}.

\subsection{Tipping at the global scale}
This global warming is unprecidented in at least the last 2000 years and has caused global temperature levels not seen in the last $125,000$ years \parencite{AR6}. This naturally leads to a question
about the nature of the change the Earth system is experiencing: will it be a smooth function of increasing \ce{CO2} or is tipping behaviour possible?

A controvertial paper,~\cite{Steffen2018}, considered the possiblity that ongoing global warming could make the Earth transition from its current current glacial-interglacial limit cycle
state into a new `hothouse' state. This state would be defined by high temperatures and sea levels, posing challenges both to humanity and the wider biosphere. This state would be reached
through biogeochemical feedbacks leading to a cascade of tipping points. This would require humanity to operate with certain `planetary boundaries' \parencite{Rockstrom2009} to avoid this possibility.

Part of the reason~\cite{Steffen2018} proved so controvertial is that there is little evidence this nonlinear response at the global scale. Global temperature rise is approximatly linear
in emitted carbon dioxide \parencite{Allen2009,Rogelj2019} and not expected to continue after emissions cease \parencite{MacDougall2020}. However certain cloud resolving models
report that at suffiently high levels of global warming stratocumulus cloud decks can break up causing \SI{8}{\kelvin} of warming globally \parencite{Schneider2019}.

\subsection{Tipping Elements}

However there is still the possibility of more regional tipping point behaviour. A foundational paper,~\cite{Lenton2008}, introduced the notion of \emph{tipping elements} which
are components of the Earth system that are `at least subcontinential in scale' ($\mathcal{O}(\SI{1000}{\kilo\meter}$) which can undergo tipping behaviour as a result of
anthropogenic influence. Lenton lead an expert elicitation of potential tipping elements in the Earth system to estimate how much warming would be needed to trigger the tipping element
and what the key uncertainties were. More recently,~\cite{ArmstrongMcKay2022} updated this study by reviewing the literature published since.  Following this, I will discuss some key tipping elements.

\subsubsection{Atlantic Meridional Overturning Circulation}
The Atlantic Meridional Overturning Circulation, known as the AMOC, is a large scale current in the ocean \parencite{}. The AMOC transports warm water polewards. This heat transport plays an
important role in the climate of, for example, northern Europe where it keeps temperatures warmer than they would be otherwise. The AMOC is driven by the salt-advection feedback, whereby
warm surface water is transported northwards where it cools and becomes more dense.  It then sinks and can make the return journey to the equator completing the circulation.

The potential for the AMOC to exhibit bistability was postulated by Stommel in a pioneering paper \parencite{Stommel1961}. He showed in a simple two box model that if the North Atlantic was
freshened then the salt-advection feedback means that meridional transport could shift to a different state. Later work \parencite{Rahmstorff1995,Hawkins2011} found multistability in
complex ocean models, when the north Atlantic was subject to a `hosing' experiment in which fresh water was added to the oceans. Other research has shown the AMOC is sensitive to the
rate of hosing \parencite{Alkhayuon2019} in a complex way \parencite{Lohmann2021} in that there is not one critical rate of hosing. By this is meant that there is no well definied critical
hosing rate but that increasing the rate will switch the rate from a dangerous one and back again.

Increased arctic precipitation, melting of the Greenland icesheet and increases in surface temperatures all act to weaken the AMOC \parencite{ArmstronMcKay2022}.
Over the past half century, there is evidence that the AMOC has weaken by around 15\% \parencite{Caesar2018} and might be the weakest in a millenium \parencite{Caesar2021}.
There is observational evidence of decreasing AMOC stability \parencite{Boers2021a,Michel2022}. Some CMIP5 models show AMOC tipping at low
levels of global warming \parencite{Drijfout2015}, although most models show only a gradual decline in the AMOC, which is helps explain why~\cite{IPCC} view AMOC shutdown as
being unlikely, although they view the modelled AMOCs as being unrealistically stable. \cite{ArmstrongMcKay2022} estimate the AMOC's theshold to be at \SI{4}{\degree\Celsius},
with a range of \SIrange{1.4}{8}{\degree\Celsius}.

Should the AMOC tip, this would have serious impacts on British agriculture \parencite{Ritchie2020a}, global climate \parencite{Jackson2015} and the carbon cycle \parencite{Bozbiyik2011}.

\subsubsection{Ice Sheets}
\subsubsection{Amazon Rainforest}
But more regional? But low likeihood, high impact.
What are tipping elements. Ice sheets. AMOC. Amazon. Permafrost? Sea Ice?

Localised tipping. Reefs, compost bomb? Fire? Clouds?

Impacts?

Tipping Cascades.

Even if these tipping elements are unlikely to tip, researchers have argued that they are `too risky to bet against' \parencite{Lenton2019a}, this is because large discontinuous change in the
Earth system could lead very severe impacts. Paul's paper. Other impacts papers? Damages dominated by the tails? ``ON MODELING AND INTERPRETING THE ECONOMICS OF
CATASTROPHIC CLIMATE CHANGE
Martin L. Weitzman''  \todo{Clean this paragraph up}
\subsection{Tipping Cascades}
\subsection{Evidence of Tipping Points from Paleoclimate}


\section{A Typology of Tipping Points}
\label{sec:tipping_typology}
\subsection{Bifurcation}
Poincare.
\subsubsection{Theory}
If $\dot{x} = f(x)$, show that $f'(x_{tip}) = 0$. Normal forms
\subsubsection{Stommel's AMOC Model}
\subsection{Noise}
\subsubsection{Theory}
Derive Kramers rate?
\subsubsection{D.O. Events}
\subsection{Rate}
\subsubsection{Theory}
3 diagrams unstable manifold
\subsubsection{Compost Bomb}
\subsection{Shock}
\subsubsection{Theory}
Calculate the distance to the boundary for a general fixed point. Suppose $x^*$ is a fixed point with basin $\mathcal{B}$. Let $d(a,b)$ be the distance from
$a$ to $b$ .Then $\Delta = \inf \{d(x,x^*) : x \in \partial\mathcal{B}\}$.
\subsubsection{An Example}
Consider the Allee Effect \parencite{Allee1932,Stephens1999}.
\begin{equation}
  \label{eq:allee_effect}
  \dot{x} = x\left(x-1\right)\left(1-\frac{x}{k}\right).
\end{equation}
Assume $k > 1$. This has equilibria at $x_1 = 0$, $x_2 = 1$ and $x_3 = k$. The equilibria $x_1,x_3$ are stable. If the system is
initially in equilibrium at $x_3$, the distance to the basin boundary is $\Delta = |x_3 - x_2| = k - 1$. Hence if the system recieves a kick of
this magnitude or greater then the system will transition to the extinction state.

%Phase tipping?

\section{Early Warning Signals}
Systems behaving 1D. Critical Slowing Down. Use of variance and
autocorrelation. Other methods. Fluctuation dissipation theorem.
Resiliance. Critical Speeding up. Hasselmann. Loreenz. Leith. 
Why is it hard to model?
\section{Non-autonomous Topics}
Overshoot. Timescales. Slow fast systems.
\chapter{Introduction}
\graphicspath{{introduction/figs}}

\section{Tipping Points}
Take a pan of water and heat it up. As its temperature rises, some of the water's properties may change. For example, the viscosity and heat flux to the
environment may change. However, the water still remains as water and these changes will vary continuously with the water temperature. However, if the water is heated beyond
a critical temperature of \SI{100}{\degreeCelsius} something more dramatic happens and the water boils away. The properties of the resulting steam are quite different
to the original water --- a \emph{qualitiative} change has occured.

It is this sort of qualitative change that this thesis shall be concerned with. Whilst the boiling of water is a phase transition \parencite{Goldenfeld1992}, I shall use the broader terminology
of \emph{tipping points} \parencite{Lenton2008} or \emph{critical transitions} \parencite{Rahmstorf1995} or \emph{abrupt changes} \parencite{Alley2003} to describe these phenomena. I shall largely stick
to the term tipping point, or just tipping, which has its origins in the idea that when an object is leant over it will return to its original position unless it is leant over sufficently far when the object
will tip over.

\subsection{Definitions}

Although it can be a vague term, the concept of a small nudge leading to a large change generally forms a part of most formal definitions of tipping points.
For example,~\cite{Lenton2008} defines the occurance of a tipping
point as when a control parameter, there is a critical valuem $\rho_{\mathrm{crit}}$ of a control parameter, $\rho$, above which any significant variation, $\delta \rho > 0$ leads to a qualitative change
$\hat{F}$ of a system feature $F$, after some observation time $T > 0$. They write this mathematically as
\begin{equation}
  \label{eq:lenton_tipping_definition}
  |F(\rho \geq \rho_{\mathrm{crit}} + \delta \rho | T) - F(\rho|T)| \geq \hat{F} > 0.
\end{equation}
More recently,~\cite{ArmstrongMcKay2022} updated this definition by requiring $\delta \rho$ to be within the natural variability of the system. They also require that changes become self perpetuating
beyond $\rho_{\mathrm{crit}}$ as a result of the asymmetry of relevant feedbacks and that the tipping leads to `substantial' impacts. The IPCC \parencite{AR6} defines tippings differently still
as `a level of change in system properties beyond which a system reorganises, often in a nonlinear manner, and does not return to the initial state even if the drivers of the change are abated'.
Others, such as~\cite{Wang2023}, inspired by~\cite{Kopp2016}, insist that tipping points should occur on fast time scales.

Each of these definitions are deficent in their own way. The IPCC definition allows for linear behaviour, although tipping is a fundamentally nonlinear phenomenon. It also insists on \emph{hysteresis},
which is the property that the system does not return to its initial state after abating the drivers, although not all examples of tipping --- including boiling water --- experience this.
The complex definition in~\cite{Lenton2008} views tipping in terms of changes in parameters of a system, and not something a system can do spontaneouslty, which excludes certain types of
tipping (see \cref{sec:tipping_typology}). The third definition, that of~\cite{ArmstrongMcKay2022}, suffers from the problems of~\cite{Lenton2008} and also requires `substantial' impacts, which
artifically restricts the abrupt shifts that can be investigated. Finally,the requirement of~\cite{Wang2023} that tipping points be realised quickly is problematic for Earth System investigations
as some of the tipping points of interest will be realised slowly.

In this thesis, I will use the term tipping point in the sense that a system undergoes a tipping point when it experiences a qualitative change in its properties. Although this is a vague definition,
it enables the term tipping point to include all dynamics of interest. I am adopting Potter Stewart's philosophy that although a tipping point might be hard to define,
`I know it when I see it' \parencite{Stewart1964}.

\subsection{Examples of Tipping}

These sorts of nonlinear changes have long facinated scientists. Physicists have investigated phase changes not just in terms of the boiling and freezing of substances but also
in magnetic materials \parencite{Ising1925,Onsager1944}, superconductivity \parencite{Landau1965} and percolation theory \parencite{Flory1941}. Each of these are different processes, but it is a
remarkable fact that near the phase transition different systems can be dynamically very similar, a phenomenon known as universality \parencite{Wilson1983}.

Tipping points are also important in biology. In medicine, tipping point theory has been used to help understand asthma attacks \parencite{Donovan2022} and sleeping dynamics \parencite{Skeldon2014}.
Tipping points have proven very important in ecology. In an influential article,~\cite{May1976}, the ecologist Robert May noted that even very simple nonlinear models have rich dynamics and are capable
of experiencing tipping points (although he did use that term). Lake Eurtrophication. Pattern Formation. Something else?



Bifurcations. Poincare.

\section{Tipping Points in the Earth System}
Trajectories of the Earth System in the Anthropocene
\subsection{Evidence of Tipping Points from Paleoclimate}
\subsection{Tipping Elements}

\section{A Typology of Tipping Points}
\label{sec:tipping_typology}
\subsection{Bifurcation}
\subsubsection{Theory}
If $\dot{x} = f(x)$, show that $f'(x_{tip}) = 0$. Normal forms
\subsubsection{Stommel's AMOC Model}
\subsection{Noise}
\subsubsection{Theory}
Derive Kramers rate?
\subsubsection{D.O. Events}
\subsection{Rate}
\subsubsection{Theory}
3 diagrams unstable manifold
\subsubsection{Compost Bomb}
\subsection{Shock}
\subsubsection{Theory}
Calculate the distance to the boundary for a general fixed point. Suppose $x^*$ is a fixed point with basin $\mathcal{B}$. Let $d(a,b)$ be the distance from
$a$ to $b$ .Then $\Delta = \inf \{d(x,x^*) : x \in \partial\mathcal{B}\}$.
\subsubsection{An Example}
Consider the Allee Effect.
\begin{equation}
  \label{eq:allee_effect}
  \dot{x} = x\left(x-1\right)\left(1-\frac{x}{k}\right).
\end{equation}
Assume $k > 1$. This has equilibria at $x_1 = 0$, $x_2 = 1$ and $x_3 = k$. The equilibria $x_1,x_3$ are stable. If the system is
initially in equilibrium at $x_3$, the distance to the basin boundary is $\Delta = |x_3 - x_2| = k - 1$. Hence if the system recieves a kick of
this magnitude or greater then the system will transition to the extinction state.

%Phase tipping?

\section{Early Warning Signals}
Systems behaving 1D. Critical Slowing Down. Use of variance and
autocorrelation. Other methods. Fluctuation dissipation theorem.
Resiliance. Critical Speeding up


\section{Non-autonomous Topics}
Overshoot. Timescales. Slow fast systems.
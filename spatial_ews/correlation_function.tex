\chapter{Correlation Function}
\label{appendix:correlation_function}

\lettrine[lines=3,loversize=0.1,findent=0.1em,nindent=0em]{C}{orrelation} functions play an important role in the analysis of spatially distributed stochastic systems.
In this appendix, I will derive the correlation function given by \cref{eq:two_point_correlation_actual}. The correlation function can be derived by taking functional
derivates of a particular functional integral.

The functional integral is given by the partition function of the system described by \cref{eq:spatial_system_y_variational}, where an additional field, $B$, has been introduced. 
Symbollicaly, the partition function is
\begin{equation}
   \label{eq:path_integral}
   Z = \int \exp\left( -\frac{2}{\sigma^2} \mathcal{H} \right) \, \mathcal{D}z,
 \end{equation}
 where
 \begin{equation}
   \label{eq:hamiltonian}
   \mathcal{H}[z] = \int_0^L \frac{1}{2}\lambda z(x)^2 + \frac{1}{2}D\left(\pdv{z}{x}\right)^2 - B(x)z(x)\dd{x}.
 \end{equation}

 To attack this integral, the tactic will be to shift to Fourier space, in which the part of the integral that survives differentiation can be factored out.
 In Fourier space, \cref{eq:hamiltonian} becomes
 \begin{equation}
   \label{eq:fourier_hamiltonian}
   \mathcal{H}[\{z_k\}] = \frac{1}{L^2}\sum_{n,m}\left(\frac{1}{2}\left(\lambda  - D nm\right) z_nz_m - B_mz_n\right)\int_0^L  e^{i(n+m)x}\dd{x}.
 \end{equation}
 The integral over $x$ is non-zero only when $m = -n$, so the sum over $m$ can be eliminated yielding
 \begin{equation}
   \label{eq:fourier_hamiltonian_contracted}
   \mathcal{H}[\{z_k\}] = \frac{1}{L}\sum_{n} \left(\frac{1}{2}\left(\lambda  + D n^2\right) z_nz_{-n} - B_{-n}z_{n}\right).
 \end{equation}
 Using the fact that $z \in \mathbb{R}$, which means that $z_n = z_{-n}^*$, \cref{eq:fourier_hamiltonian_contracted} can be simplified to
 \begin{equation}
   \label{eq:fourier_hamiltonian_contracted_mod}
   \mathcal{H}[\{z_k\}] = \frac{1}{L}\sum_{n} \left(\frac{1}{2}\left(\lambda  + D n^2\right) \left|z_n\right|^2- B_{-n}z_{n}\right).
 \end{equation}
 Setting $\phi_k = z_k - \frac{B_k}{Dk^2 + \lambda}$ and completing the square means $\mathcal{H}$ is now
 \begin{equation}
   \label{eq:fourier_phi}
   \mathcal{H}[\{\phi_k\}] =  \frac{1}{L}\sum_{n} \left(\frac{1}{2}\left(\lambda + Dn^2\right) \left|\phi_n\right|^2 - \frac{1}{2}\frac{\left|B_n\right|^2}{\lambda + Dn^2}\right).
 \end{equation}
 This can be written as $\mathcal{H}[\{\phi_k\}] = \mathcal{H}_0[\{\phi_k\}] - \frac{1}{2L}\sum_{n}\frac{\left|B_n\right|^2}{\lambda + Dn^2}$, so that \cref{eq:path_integral} becomes
 \begin{equation}
   \label{eq:path_integral_factored}
   Z = \int \prod_{k>0} \exp \left(-\frac{2}{\sigma^2} \mathcal{H}_0[\{\phi_k\}]\right) \dd{\phi_k}\exp\left(\frac{1}{\sigma^2} \frac{1}{L}\sum_{n}\frac{\left|B_n\right|^2}{\lambda + Dn^2}\right)
 \end{equation}
 or
 \begin{equation}
   \label{eq:path_integral_factored_Z0}
   Z = Z_0 \exp\left(\frac{1}{\sigma^2} \frac{1}{L}\sum_{n}\frac{\left|B_n\right|^2}{\lambda + Dn^2}\right).
 \end{equation}
 Inverting $B_k$ back into real space gives
 \begin{equation}
   \label{eq:path_integral_factored_Z0_real}
   Z = Z_0 \exp\left(\frac{1}{\sigma^2} \iint B(x_1) g(x_1 - x_2) B(x_2) \dd{x_1}\dd{x_2}\right).
 \end{equation}
 where
 \begin{equation}
   \label{eq:G_from_Z}
   g(x_1 - x_2) =  \frac{1}{L}\sum_{n} \frac{e^{in(x_1-x_2)}}{\lambda + Dn^2}. 
 \end{equation}
 
 This sum can be evaluated by replacing it with an integral, assuming $L$ is large enough:
 \begin{equation}
   \label{eq:G_as_integral}
   g(x) = \frac{1}{2\pi}\int_{-\infty}^{\infty}  \frac{e^{ikx}}{\lambda + Dk^2} \dd{k} = \frac{1}{2} \frac{\xi}{D} e^{-x/\xi},
 \end{equation}
 hence using \cref{eq:two_point_from_partition},
 \begin{equation}
   \label{eq:G_from_G}
   G(x) = \frac{1}{2}\sigma^2g(x) =  \frac{1}{4} \sigma^2 \frac{\xi}{D} e^{-x/\xi}.
 \end{equation}
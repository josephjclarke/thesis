\chapter{Spatial Early Warning Indicators}
\graphicspath{{spatial_ews/figs/}}

\section{Fast and Slow Tipping Elements}
An open question in the theory of Early Warning Signals is how to make use of them when the control parameter
changes quickly relative to the timescale of the system\cite{VanderBolt2021}\todo{I should read this paper properly}.
To be more formal about this, consider the system
\begin{equation}
  \label{eq:system_with_a_timescale}
  \dv{y}{t} = \frac{f(y,rt)}{T}
\end{equation}
where $y$ is the system state, $T$ is a characteristic timescale of the system and $r$ is the rate the system is linearly forced
at. Suppose further than there is a tipping point at time $t = t^*$ (i.e. when the forcing reaches a magnitude of $rt^*$). By considering
the ratio of the system to forcing timscale we can define a parameter $\epsilon = rT$ and introduce a rescaling of time $t=Ts$
so that \cref{eq:system_with_a_timescale} becomes
\begin{equation}
  \label{eq:system_with_a_timescale_rescaled}
  \dv{y}{s} = f(y,\epsilon s).
\end{equation}

There is an important limit to be considered here. Suppose $r$ is very small --- in other words the system is forced very slowly --- so that we
can take $\epsilon\rightarrow 0$ in such a way that $\epsilon s \rightarrow \mu$ where $\mu$ is a constant. We can now view this as a bifurcation
problem and can then apply all the usual tools to detect approaching tipping points\cite{scheffer2009}.

This limit of slow forcing can also be viewed as the limit of a system with a very rapid timescale. Hence tipping point assossiated to the system
this limit applies to (whether because it is slowly forced or rapidly responding) is known as a \emph{fast tipping point}.

Alternatively we can look at so called \emph{slow tipping elements} in which $\epsilon$ is not small. It is difficult to get good Early Warning Signals
using the temporal variance and autocorrelation\cite{VanderBolt2021}. Unfortunately these sorts of systems are widespread in climate change science.
Recall that the rate of global warming is fast, on the order of \SI{1}{\kelvin} per century\cite{Osborn2021}, a rate unprecidented in
a long time\todo{find a good citation for this}. Many important tipping elements have timescales of centuries or longer\todo{cite timescales}.
This implies that for contemporary climate change $\epsilon$ is unlikely to be small. This motivates the development of technqiues that enable
larger values of $\epsilon$ to be probed.

Part of the problem with these slow tipping elements is that the forcing changes appreciably over the window used to calculate early warning signals.
To get around this we might try to make `instantaneous' measurements of its variance and autocorrelation. One way to do this would be to calculate these
statistics over space instead of over time.

\section{Prior Work}
\todo{What is known about spatial EWS already?}
\chapter{Spatial Early Warning Signals}
\graphicspath{{spatial_ews/figs}}

\section{Fast and Slow Tipping Elements}
An open question in the theory of Early Warning Signals is how to make use of them when the control parameter
changes quickly relative to the timescale of the system\cite{VanderBolt2021}\todo{I should read this paper properly}.
To be more formal about this, consider the system
\begin{equation}
  \label{eq:system_with_a_timescale}
  \dv{y}{t} = \frac{f(y,rt)}{T}
\end{equation}
where $y$ is the system state, $T$ is a characteristic timescale of the system and $r$ is the rate the system is linearly forced
at. Suppose further than there is a tipping point at time $t = t^*$ (i.e.\ when the forcing reaches a magnitude of $rt^*$). By considering
the ratio of the system to forcing timescale we can define a parameter $\epsilon = rT$ and introduce a rescaling of time $t=Ts$
so that \cref{eq:system_with_a_timescale} becomes
\begin{equation}
  \label{eq:system_with_a_timescale_rescaled}
  \dv{y}{s} = f(y,\epsilon s).
\end{equation}

There is an important limit to be considered here. Suppose $r$ is very small --- in other words the system is forced very slowly --- so that we
can take $\epsilon\rightarrow 0$ in such a way that $\epsilon s \rightarrow \mu$ where $\mu$ is a constant. We can now view this as a bifurcation
problem and can then apply all the usual tools to detect approaching tipping points\cite{scheffer2009}.

This limit of slow forcing can also be viewed as the limit of a system with a very rapid timescale. Hence tipping point associated to the system
this limit applies to (whether because it is slowly forced or rapidly responding) is known as a \emph{fast tipping point}.

Alternatively we can look at so called \emph{slow tipping elements} in which $\epsilon$ is not small. It is difficult to get good Early Warning Signals
using the temporal variance and autocorrelation\cite{VanderBolt2021}. Unfortunately these sorts of systems are widespread in climate change science.
Recall that the rate of global warming is fast, on the order of \SI{1}{\kelvin} per century\cite{Osborn2021}, a rate unprecedented in at least
the last 2000 years\cite{AR6}. Many important tipping elements have timescales of centuries or longer.
This implies that for contemporary climate change $\epsilon$ is unlikely to be small. This motivates the development of techniques that enable
larger values of $\epsilon$ to be probed.

Part of the problem with these slow tipping elements is that the forcing changes appreciably over the window used to calculate early warning signals.
To get around this we might try to make `instantaneous measurements of its variance and autocorrelation. One way to do this would be to calculate these
statistics over space instead of over time.

Spatial early warning signals have been studied before. \cite{Donangelo2010} compared spatial and temporal early warning signals, finding that the spatial variance
can give a better early warnign than the temporal variance. Another study, \cite{Kefi2014}, found rising memory, variability and changes to patchiness were all
spatial indicators of an upcoming transition. A range of studies have applied spatial early warnings to ecological problems\cite{Carpenter2010,Dakos2011,Guttal2009}.
Spatial early warning signals have even been applied to observational data\cite{Tirabassi2023,Kefi2007,Eby2017}. It should be noted that recent high profile applications to
the earth system have all involved temporal rather than spatial indicators\cite{Boulton2022,Boers2021,Boers2021a}.\todo{I should read these studies\ldots}

\section{The System}
In order to investigate spatial early warning signals we will investigate a specific system. We will choose this system
to be generic enough that broader conclusions can be drawn. We will make the system as simple as possible to make it more
likely to give understandable results.

Making the hypothesis that near a tipping point many systems are governed by coherent dynamics leads us to introduce one dependent
variable $y$. In one dimension there is only one generic type of bifurcation\cite{Thompson1994}, the saddle node. Furthermore
near a saddle node bifurcation, all systems show qualitatively similar behaviour\cite{guckenheimer2013} so the dynamics of $y$ should
involve the lowest order terms which give a saddle node, namely $\dot{y} = y - y^3/3 - \mu$, where $\mu$ is a control parameter.

We need to take into account the spatial nature of the problem. We will restrict attention to a 1 dimensional
periodic domain. To couple in space we use the well studied diffusive coupling which acts to smooth out the value of $y$.
Such a form has found numerous uses in climate and ecological models.
For example, it has been used in energy balance models for the global temperature\cite{Ghil1976}, ecosystem pattern formation\cite{Gowda2014}
and in continuum models of the compost bomb\cite{Clarke2021}.

We will introduce noise into the system and assume the control parameter changes linearly with time. All of these considerations motivates studying
\begin{subequations}
\label{eq:spatial_system}
  \begin{align}
    \pdv{y}{t} &= y - \frac{1}{3}y^3 - \mu + D\pdv[2]{y}{x} + \sigma\zeta \label{eq:spatial_system_y} \\
    \dv{\mu}{t}&= \epsilon \label{eq:spatial_system_mu}
  \end{align}
\end{subequations}
where $\zeta$ is a white noise with mean $\left\langle \zeta\left(x,t\right) \right\rangle = 0$ and covariance $\left\langle \zeta\left(x,t\right)\zeta(x',t')\right\rangle = \delta(x-x')\delta(t-t')$.
It therefore follows that $\sigma$ is the magnitude of the noise.

We set the size of the domain to be $L$. The variable $\mu$ linearly increases between $\mu = 0$ and $\mu = 1$ and so the system will undergo a saddle node
bifurcation at $\mu = 2/3$.

Alternatively, we can view the system variationally\footnote{A brief review of the concepts of the variational derivative are presented in \cref{appendix:variational_calculus}.}.
To do this we introduce the `energy'
\begin{equation}
  \label{eq:spatial_system_energy}
  \mathscr{E} = \int_0^L -\frac{1}{2}y^2 + \frac{1}{12}y^4 + \mu y + \frac{1}{2}D\left(\pdv{y}{x}\right)^2\,\mathrm{d}{x}.
\end{equation}

This gives further justification for using a diffusive coupling in the following sense. The lowest order term that one could include in $\mathscr{E}$ involving
$y$ and its derivatives that is isotropic in space and invariant under $y \rightarrow -y$ is the derivative term in \cref{eq:spatial_system_energy}. In that sense a diffusive term is the simplest
and, if spatial gradients are not too strong, the most relevant form of spatial coupling.

Armed with $\mathscr{E}$ we can rewrite \cref{eq:spatial_system_y} as
\begin{equation}
  \label{eq:spatial_system_y_variational}
    \pdv{y}{t} = -\fdv{\mathscr{E}}{y(x)}  + \sigma\zeta, 
\end{equation}
which is a Time Dependent Ginzburg Landau equation\cite{goldenfeld1992}.

\section{Statistics}
We will compare the autocorrelation and variance of \cref{eq:spatial_system} calculated over space and over time. We will now
make precise what that means. For a summary see \cref{tab:ews_space_time_definition}.

\subsection{Spatial Statistics}

We will define the average of a quantity calculated from \cref{eq:spatial_system}, $A(t) = A(y(x,t),x)$ in the obvious way
\begin{equation}
  \label{eq:definition_of_average}
  \langle A \rangle = \frac{1}{L}\int_0^L A(y(x,t),x) \,\mathrm{d}x.
\end{equation}
We can now define the spatial variance of $y$ at time $t$ as
\begin{equation}
  \label{eq:spatial_variance}
  \sigma_s(t)^2 = \langle y^2 \rangle - \langle y \rangle.
\end{equation}
We will define the autocorrelation over space as
\begin{equation}
  \label{eq:spatial_autocorrelation}
  \alpha_s(t) = \frac{\langle y(t,x)y(t+\Delta t,x) - \langle y(t,x)^2 \rangle }{\sigma_s(t)^2}.
\end{equation}
The quantity $\Delta t$ is the lag of the autocorrelation. However \cref{eq:spatial_system} will ultimately be solved numerically and
in everything that follows we will set $\Delta t$ to the timestep, so that $\alpha_s$ is the lag-1 autocorrelation.

\subsection{Temporal Statistics}
When working over time, we will convert the spatio-temporal data to temporal data by first averaging in space to get $\left\langle y \right\rangle$.
To calculate the early warning indicators over time, we must define averaging in time. Consider the quantity $B(t)$. We will
define the temporal average as
\begin{equation}
  \label{eq:definition_of_temporal_average}
  \overline{B} = \frac{1}{T}\int_0^TB(t)\,\mathrm{d}t
\end{equation}
where the average is taken over some suitable window of length $T$.

When computing early warning indicators, we must first `detrend' the time series. We denote the detrended time series of $\langle y \rangle$ as $z$.
The the variance and autocorrelation, defined over a window of length $\tau_w$ are
\begin{equation}
  \label{eq:temporal_variance}
  \sigma_t^2 = \overline{z^2} - \overline{z}^2 
\end{equation}
and
\begin{equation}
  \label{eq:temporal_autocorrelation}
  \alpha_t = \frac{\overline{z(t)z(t+\Delta t)}}{\sigma_t^2}.
\end{equation}
\begin{table}
  \centering
  \begin{tabular}{llll}
    \toprule
    Quantity        & Domain & Symbol        & Definition \\
    \midrule
    Variance        & Space  & $\sigma_s^2$  & $\langle y^2 \rangle - \langle y \rangle$ \\
    \rule{0pt}{4ex}    
                    & Time   & $\sigma_t^2$  & $\overline{z^2} - \overline{z}^2$ \\
    \rule{0pt}{4ex}    
    Autocorrelation & Space  & $\alpha_s$    & $\frac{\left\langle y\left(t,x\right)y\left(t+\Delta t,x\right)\right\rangle - \left\langle y\left(t,x\right)^2 \right\rangle}{\sigma_s(t)^2}$ \\
    \rule{0pt}{4ex}    
                    & Time   & $\alpha_t$    &  $\frac{\overline{z(t)z(t+\Delta t)}}{\sigma_t^2}$ \\
    \bottomrule
  \end{tabular}
  \caption[Definition of spatial and temporal early warning signals]{The definition of the early warning indicators, over space and over time}
  \label{tab:ews_space_time_definition}
\end{table}

\section{Numerical Results}
\subsection{Numerical Method}
To solve \cref{eq:spatial_system}, we discretise the domain into $N = 100$ grid
points and 
calculate the diffusive term with finite differences. We denote the value of $y$ at the $k$th gridpoint by $y_k$. We integrate forward in time by 
using an implicit Euler method, where we solve the resulting nonlinear equation using 
MINPACK's HYBRD method accessed through SciPy\cite{Virtanen2020}. To avoid excessively long integrations when $\epsilon$ is small,
we set the timestep $\delta t = 0.001/\epsilon$ and integrate until $t=1/\epsilon$. We
calculate all early warning signals from $\epsilon t=0$ until $\epsilon t=2/3$, which is the 
time of the tipping point. When working over time, we choose our window size to 
contain 500 data points (i.e.\ half the time series),
so that the window length is $\tau_w = 1/(2\epsilon)$. We perform a quadratic detrend in these windows.
To stand a chance of getting good early warning
signals in time we require that $\epsilon < 1/\tau_w$, which is always satisfied.
Systems are initialised in equilibrium and spun up for 1000 time steps.
\subsection{Two Limits}
Before fully exploring the effect of $\epsilon$ and $D$ on early warning signals, we will look at two important limits.    
\subsubsection{Uncoupled Limit}
We will  begin by examining the case when $D = 0$, for large and small values of $\epsilon$. Consider \cref{fig:uncoupled_timeseries}.
\begin{figure}
  \centering
  \includegraphics[width=\textwidth,keepaspectratio]{uncoupled_variance}
  \caption[Early warning signals in the uncoupled limit]{Early warning signals when $D = 0$. The left column shows the slow forcing case and the right column shows the fast forcing.
    The top row shows the individual trajectories with the mean trajectory shown in green. The black curves are the quasi-static equilibria.
    In the second row the we plot $\sigma_s^2$ in black and $\sigma_t^2$ in red. In the bottom row we calculate $\sigma_t^2$ from the individual
    $y_k$ values rather than the mean and calculate the Kendall $\tau$ value for each and plot a histogram of the results. The mean $\tau$ values
                are 0.6 and 0.04 for the slow and fast case respectively.}
  \label{fig:uncoupled_timeseries}
\end{figure}
 The left column shows the small $\epsilon$ case. This is the case
    where we expect temporal early warning signals to work well. For this value of $\epsilon$ the system
    transitions to its new state near the time of the bifurcation and its variance calculated over
    space or calculated over time both clearly increase near the bifurcation point. Furthermore
    calculating the statistics for each the individual grid points shows that most grid points 
    experience a rise in variance over time as well.

    For the larger $\epsilon$ case the results are different. The system has not yet transitioned to
    its new state even by the end of the simulation. We do not see a clear rise in $\sigma_t^2$ near the bifurcation point. Furthermore looking at the individual grid points
    we do not see a coherent warning of the upcoming transition. However there is a very clear rise in $\sigma_s^2$ before the bifurcation, demonstrating that spatial early warning
    signals are superior in this case.

    \subsubsection{Slowly Forced Limit}
    \Cref{fig:coupled_timeseries} is similar to \cref{fig:uncoupled_timeseries} expect we now work in the limit of slow forcing ($\epsilon = 0.01$) but
    a non-zero coupling in space. We choose $D = 1$ and $D = 10^5$. For low $D$ and slow forcing we expect early warning signals to work both when calculated spatially or
    temporally. At higher $D$ we expect the correlations between the grid points to be so great that there will no longer be any variability between them, and so no detection
    of spatial early warning signals should be possible. This is precisely what we see in \cref{fig:coupled_timeseries} where for $D=10^5$ the system acts like a single grid point.
    \begin{figure}
      \centering
      \includegraphics[width=\textwidth,keepaspectratio]{coupled_variance}
      \caption[Early Warning Signals in the slowly forced limit]{Early warning signals when $\epsilon = 0.01$. The left column shows 
                the weakly coupled case, the right shows what happens in the strongly coupled (large $D$) case. 
                The top row shows the quasi-static equilibria (black) the individual
                trajectories of the grid boxes (grey) and the domain average (green).
			      The variance is plotted in the second row, calculated over space 
                (black) and over time (red). The bottom row shows a histogram of the Kendall $\tau$ values calculated over time for each $y_k$.}
              \label{fig:coupled_timeseries}
    \end{figure}
    \subsection{Exploring the $\epsilon$ and $D$ parameter space}
    Now that we have examined the two limits, it will be worthwhile to more explore the parameter space more fully.
    We will investigate points in the parameter plane defined by $(\epsilon,D) \in [10^{-2},10^1] \times [10^{-7},10^{7}]$.
    We sample 50 points in each direction, geometrically spaced, giving $2500$ total points.

    We can determine if an early warning indicator is increasing or decreasing by calculating the Kendall's $\tau$ for the data.
    A positive value implies an increasing trend\cite{Wilks2011}. We can assess the reliability of the warning by repeating the numerical
    experiment 100 times (with different realisations of the noise) and calculating a distribution of Kendall's $\tau$. If the signal-to-noise ratio
    (SNR) defined as the ratio of the mean to the standard deviation is larger than 1, then we cannot reliably expect and early warning.

    We calculate this signal-to-noise ratio in \cref{fig:parameter_plane} for temporal and spatial early warning statistics. We see that only for
    certain regions of the parameter plane do we get an early warning of the upcoming tipping point. We note that there is a complementarity between spatial
    and temporal early warning signals. We see that for rapidly forced systems we get good early warning signals in space as long the
    the coupling in space is not too strong. Conversely, for strongly coupled systems we get 
    a good early warning in time as long the forcing is slow enough. We observe the strongly
    coupled and rapidly forced region does not give good early warning signals with either method.

    \begin{figure}
      \centering
      \includegraphics[width=\textwidth,keepaspectratio]{parameter_plane}
      \caption[The quality of early warning signals in the $epsilon$ and $D$ plane]{Red indicates regions of parameter space where the EWS are reliable, blue
        where they are not. In the dashed and dotted lines we plot the lines $D=L^2$ and
        $\epsilon(1+\frac{4\pi^2}{L^2} D) = 1$.}
      \label{fig:parameter_plane}
    \end{figure}
    \section{Scaling Arguments}
    \label{sec:scaling_arguments}
    
    In this section we will explain why the early warning signals only work in certain regions of \cref{fig:parameter_plane}.
    \subsection{Temporal Early Warning Signals}
    Consider the equilibrium solution to \cref{eq:spatial_system_y_variational} (i.e.\ at fixed $\mu$). This will correspond to a minimum
    of \cref{eq:spatial_system_energy}. Noting that \cref{eq:spatial_system_energy} is a strictly increasing function of $\pdv*{y}{x}$ it must be the case
    that the equilibrium solution is uniform in $x$ and hence independent of $D$.

    This motivates setting $y = y_{\mathrm{eq}} + z$ where $y_{\mathrm{eq}}$ is the equilibrium solution to \cref{eq:spatial_system_y_variational}. Temporal
    early warning signals are based on $\langle y \rangle$ so we will compute its evolution equation by averaging \cref{eq:spatial_system} in space.
    This gives
    \begin{align*}
      \dv{\langle y \rangle}{t} &= \langle y \rangle - \frac{1}{3} \left\langle y^3 \right\rangle - \mu + D \left\langle \pdv[2]{y}{x} \right\rangle + \sigma \eta \\
                               &= \langle y \rangle - \frac{1}{3} \left\langle y^3 \right\rangle - \mu + \sigma \eta
    \end{align*}
    where $\eta$ is white noise in time only and the diffusive term vanishes due to the periodic boundary conditions. Although there is no explicit dependence on $D$ here,
    this equation still implicitly depends on $D$ through the nonlinear averaged term $\left\langle y^3 \right\rangle$.

    We can now determine the evolution equation for $z = y - y_{\mathrm{eq}}$  as
    \begin{equation}
      \label{eq:averaged_linear_perturbation_evolution}
      \dv{\langle z\rangle}{t} = \left(1 - y_{\mathrm{eq}}^2\right)\left\langle z \right \rangle + \sigma \eta  + \mathcal{O}\left(z^2 \right)
  \end{equation}
  where we have used the fact that $y_{\mathrm{eq}}$ satisfies \cref{eq:spatial_system_y} and Taylor expanded to first order in $z$. This has no dependence on
  $D$ and hence we expect the temporal early warning signals to be independent of $D$.

  To get good early warning signals we shall require $\langle z\rangle$ to be small so that the linearisation in
  \cref{eq:averaged_linear_perturbation_evolution} is valid, but as $z$ is $\mathcal{O}\left(\epsilon^{1/3}\right)$\cite{Berglund2006},
  we expect good temporal early warning signals when $\epsilon^{1/3} \ll 1$.

  \subsection{Spatial Early Warning Signals}
  Analysing the system spatially is aided by recognising the connection between \cref{eq:spatial_system_y_variational} and Landau theory for the Ising
  model\cite{goldenfeld1992}, where
  $y$ can be viewed as a spin and $\mu$ as an external applied magnetic field.

  This theory gives the correlation length as $\xi = \sqrt{D}$. In order to get good statistics, we need to be sampling over a region
  large compared to $\xi$. Hence we only expect to get good early warnings when $D \ll L^2$, where $L$ is the size of the domain.

    Furthermore, the spatial dimension introduces extra timescales into the system, associated with
    fluctuations of a particular spatial scale. Upon linearising \cref{eq:spatial_system_y} and Fourier decomposing, we find
    $\pdv*{z_k}{t} = -z_k/\tau_k + \sigma\eta_k$ where $z_k$ refers to the Fourier mode of the fluctuation $z$ associated with wavenumber $k$ and 
    these fluctuations relax towards 
    equilibrium on a timescale $\tau_k$. The inverse relaxation time
    for modes with wavenumber $k$ is $\tau_k^{-1} = 1 + Dk^2$. It is these fluctuations that will ultimately
    give us early warning signals, so they cannot be too short lived. Hence requiring that 
    $\tau_k \gg \epsilon$\todo{(do I buy this?)}
    for all $k$ up to the scale of the simulation ($k = 2\pi/L$) implies
    that we need $\epsilon(1+\frac{4\pi^2}{L^2} D)\ll 1$. Hence we see that for $D$ small enough, we 
    expect early warnings in space for larger values of $\epsilon$ than for early warnings in time.

    \subsection{Comparison With Numerical Experiments}
    \Cref{fig:parameter_plane} implies the critical $\epsilon$ above which temporal early warning signals fail is $\epsilon_c \approx 0.04$.
    Performing the computing $\epsilon_c^{1/3} = 0.34$ shows that this is indeed compatible with our theory that which requires that
    $\epsilon^{1/3} \ll 1$. We also plot the lines $D = L^2$ and $\epsilon\left(1 + 4\pi^2 D/L^2\right) = 1$ onto \cref{fig:parameter_plane}.
    These lines bound the region where spatial early warning signals work and so these results are compatible with our theory.
    \section{Discussion}
     We have seen that calculating early warning signals over space allows the user to get reliable
    signals in a previously inaccessible parameter regime (for larger values of $\epsilon$).  As well
    as having this advantage we note further that calculating early warning signals over space avoids 
    the problems of detrending which are inherent to the method of calculating early warning signals over
    space.

    There is a nice complementarity however between early warnings in space and time. For fast forcing
    and weak coupling in space, early warnings in space work but those in time do not. For slow forcing
    but strong spatial coupling then early warnings in time work but those in space do not. If the
    coupling is weak but the forcing is slow both methods work. There is still an inaccessible parameter
    region, for fast forcing and strong spatial coupling. This is illustrated schematically in figure
    \cref{fig:idealised_plot}.
    \begin{figure}
      \centering
      \includegraphics[width=\textwidth,keepaspectratio]{idealised_plot}
      \caption[A schematic showing the complementarity of spatial and temporal early warning signals]{Figure 4: A schematic illustrating the complementarity of early warning signals in space and time.
        The lines are drawn according to the scalings given in \cref{sec:scaling_arguments}}
        \label{fig:idealised_plot}
    \end{figure}

\section{Conclusion}

Temporal early warning signals for tipping points are challenging, not least because we are often interested in rapidly forced systems. In this chapter
We have advanced the hypothesis that spatial early warning signals may be useful in these more rapidly forced systems.
We have found that spatial early warning signals can be used for more rapidly forced systems, as \cref{fig:uncoupled_timeseries,fig:parameter_plane}
illustrate.

We have found that their usefulness decreases as the spatial interactions increase\todo{Can we estimate these parameters for real systems?}.
In particular we see a dichotomy between temporal early warning signals (which work well in strongly spatially coupled and slowly temporally forced systems)
and spatial early warning signals (which work well in weakly spatially coupled and strongly temporally forced systems).

We note that there is still an inaccessible parameter region --- systems forced strongly in time and strongly coupled in space. There are also further limitations
to our approach.  We have only considered one functional form of coupling, but others are possible. Secondly, we have made
assumptions that space is homogeneous, but if $D$ is itself a function of $x$ then this could introduce problems.

Overall we feel that spatial early warning signals, used in conjunction with temporal ones, can provide a useful tool to detect upcoming tipping points.
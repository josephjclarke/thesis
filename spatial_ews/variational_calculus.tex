\chapter{Variational Calculus}
\label{appendix:variational_calculus}

In this appendix, we give a brief summary of some of the results involving the \emph{functional} or \emph{variational} derivative.
There are no original results in this appendix and it draws on \cite{Lancaster2014,goldenfeld1992}. Both for reasons of clarity and because we are not
working rigorously we will adopt the notation favoured by physicists.

\section{Functionals}
A function is a map between numbers. For example the function $f\colon \mathbb{R} \rightarrow \mathbb{R}$
\begin{equation}
  \label{eq:example_function}
  f(x) = x^2
\end{equation}
associates the input number $2$ to the output number $4$.

However \emph{functionals} associate functions to numbers. They are ubiquitous in mathematics and physics. For example
the definite integral
\begin{equation}
  \label{eq:example_functional_definite_integral}
  I[f] = \int_{-\infty}^{\infty} f(x) \, \mathrm{d}x
\end{equation}
is a functional of $f$. It associates the function $e^{-x^2}$ to the number $\sqrt{\pi}$.
So too is the evaluation functional
\begin{equation}
  \label{eq:example_functional_evaluation}
  E_y[f] = f(y).
\end{equation}
Most famous of all is the \emph{action} of a system
\begin{equation}
  \label{eq:example_functional_action}
  S[q] = \int_0^t L(q(t'),\dot{q}(t'),t')\,\mathrm{d}t'.
\end{equation}

\section{Derivatives}
The definition of a derivative is well known:
\begin{definition}[Derivative of a Function]
  \label{definition:derivative}
  Let $f\colon \mathbb{R} \rightarrow \mathbb{R}$ be a function. Then the derivative of $f$ is
  \begin{equation}
    \label{eq:definition_of_derivative}
    \dv{f}{x} = \lim_{\epsilon\rightarrow 0} \frac{f(x + \epsilon) - f(x)}{\epsilon}
  \end{equation}
  if the limit exists.
\end{definition}
In other words, the derivative measures how much a function changes when its input changes `a little bit'.

Similarly, we can ask what happens to a functional when its input is changed `a little bit'. We can therefore
define the functional derivative (also known as a variational derivative) as follows.
\begin{definition}[Functional Derivative]
  \label{definition:functional_derivative}
  Let $J$ be a functional of $f$ which is in turn a function of $x \in \mathbb{R}$. Then the functional derivative is
  \begin{equation}
    \label{eq:definition_of_functional_derivative}
    \fdv{J}{f(x)} = \lim_{\epsilon\rightarrow 0} \frac{J[f(x') + \epsilon \delta(x - x')] - J[f(x')]}{\epsilon}
  \end{equation}
  if the limit exists. Here, $\delta(x)$ is the Dirac Delta function.
\end{definition}

As an example, consider the functional derivative of \cref{eq:example_functional_definite_integral}. From \cref{definition:functional_derivative}
we have
\begin{equation*}
  \fdv{I}{f(x)} = \lim_{\epsilon\rightarrow 0} \frac{1}{\epsilon} \left(\int_{-\infty}^{\infty} f(x') + \epsilon \delta\left(x-x'\right)\,\mathrm{d}x' 
      - \int_{-\infty}^{\infty} f(x')\,\mathrm{d}x'\right),
\end{equation*}
which can be simplified to give
\begin{equation*}
  \fdv{I}{f(x)} = \lim_{\epsilon\rightarrow 0}\frac{1}{\epsilon}\int_{-\infty}^{\infty} \epsilon \delta(x-x')\,\mathrm{d}x' = 1.
\end{equation*}

\section{Useful Results}
We now prove two chain rule type results for functional derivatives.
\begin{theorem}
  \label{theorem:chain_rules}
  Let $f,g\colon\mathbb{R}\rightarrow\mathbb{R}$ be suitable functions. Then
  \begin{equation}
    \label{eq:chain_rule_functionals}
    \fdv{f(x)} \int_a^b g\left(f\left(y\right)\right)\,\mathrm{d}y = g'\left(f\left(x\right)\right)
  \end{equation}
  and
  \begin{equation}
    \label{eq:chain_rule_derivatives_functional}
    \fdv{f(x)} \int_a^b g\left(f'\left(y\right)\right)\,\mathrm{d}y = -\dv{x}\left(\dv{g\left(f'\left(x\right)\right)}{f'}\right)
  \end{equation}
  where conventional derivatives have been denoted with primes.
\end{theorem}
\begin{proof}
  We define
  \begin{equation*}
    J[f] = \int_a^b g\left(f\left(y\right)\right)\,\mathrm{d}y
  \end{equation*}
  Beginning with the first claim using \cref{definition:functional_derivative}:
  \begin{align*}
    \fdv{J}{f(x)}  &= \lim_{\epsilon\rightarrow 0} \frac{1}{\epsilon} \left(\int_a^b g\left(f(y) + \epsilon \delta\left(y-x\right)\right)\,\mathrm{d}y
                                                                     - \int_a^b g\left(f\left(y\right)\right)\mathrm{d}y \right)\\
    &=  \lim_{\epsilon\rightarrow 0} \frac{1}{\epsilon} \left(\int_a^b g\left(f(y)\right) + \epsilon \delta\left(y-x\right)g'\left(f\left(y\right)\right)\,\mathrm{d}y
                                                                     - \int_a^b g\left(f\left(y\right)\right)\,\mathrm{d}y\right)
  \end{align*}
  where the function $g$ has been Taylor expanded to first order. We ignore the second order terms as they will vanish in the limit.
  Taking the limit gives
  \begin{align*}
    \fdv{f(x)} \int_a^b g\left(f\left(y\right)\right)\,\mathrm{d}y &= \int_a^b \delta(y - x) g'(f(y))\,\mathrm{d}y \\
                                                                   &= g'(f(x)).                                                              
  \end{align*}
  Defining
  \begin{equation*}
    J[f] =  \int_a^b g\left(f'\left(y\right)\right)\,\mathrm{d}y,
  \end{equation*}
  we proceed similarly for the second claim:
  \begin{align*}
    \fdv{J}{f(x)}  &= \lim_{\epsilon\rightarrow 0} \frac{1}{\epsilon}\left(\int_a^b g\left(\pdv{y}\left(f(y) + \epsilon \delta (y-x)\right)\right)\, \mathrm{d}y- \int_a^b g \left( f' \left( y \right) \right)\mathrm{d}y\right)       \\
                                                                    &= \lim_{\epsilon\rightarrow 0} \frac{1}{\epsilon}\left(\int_a^b g(f'(y)) + \epsilon \delta' (y-x)\dv{g(f'(y))}{f'}\, \mathrm{d}y- \int_a^b g(f'(y))\,\mathrm{d}y\right) \\
                                                                    &= \int_a^b \delta' (y-x)\dv{g(f'(y))}{f'}\, \mathrm{d}y \\
    &= - \dv{x} \dv{g(f'(x))}{f'}
  \end{align*}
  after integrating by parts.
\end{proof}

An important special case of \cref{theorem:chain_rules} is
\begin{equation}
  \label{eq:derivative_of_squared_derivative}
  \fdv{f(x)}\int_a^b \left(\pdv{f}{y}\right)^2\,\mathrm{d}y = - 2 \pdv[2]{f}{x}.
\end{equation}
Note the minus sign.
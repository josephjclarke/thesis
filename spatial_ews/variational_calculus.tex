\chapter{Variational Calculus}
\label{appendix:variational_calculus}

\lettrine[lines=3,loversize=0.1,findent=0.1em,nindent=0em]{S}{ome} of results involving the \emph{functional} or \emph{variational} derivative will be summarised in this section.
This section largely follows~\cite{Lancaster2014} and~\cite{Goldenfeld1992}. The notation adopted here will be the one used by physicists
both for reasons of clarity and also to emphasise the non-rigorous presentation.

\section{Functionals}
A function is a map between numbers. For example the function $f\colon \mathbb{R} \rightarrow \mathbb{R}$
\begin{equation}
  \label{eq:example_function}
  f(x) = x^2
\end{equation}
associates the input number $2$ to the output number $4$.

However \emph{functionals} associate functions to numbers, they are often denoted with square brackets. For example
the definite integral
\begin{equation}
  \label{eq:example_functional_definite_integral}
  I[f] = \int_{-\infty}^{\infty} f(x) \, \mathrm{d}x
\end{equation}
is a functional of $f$. It associates the function $e^{-x^2}$ to the number $\sqrt{\pi}$.

An important functional is the action of a system
\begin{equation}
  \label{eq:example_functional_action}
  S[q] = \int_0^t L(q(t'),\dot{q}(t'),t')\,\mathrm{d}t'.
\end{equation}
where $L$ is known as the Lagrangian.


\section{Derivatives}
The definition of a derivative is well known:
\begin{definition}[Derivative of a Function]
  \label{definition:derivative}
  Let $f\colon \mathbb{R} \rightarrow \mathbb{R}$ be a function. Then the derivative of $f$ is
  \begin{equation}
    \label{eq:definition_of_derivative}
    \dv{f}{x} = \lim_{\epsilon\rightarrow 0} \frac{f(x + \epsilon) - f(x)}{\epsilon}
  \end{equation}
  if the limit exists.
\end{definition}
More intuitively, the derivative measures how much a function changes when its input changes by a small quantity.

Similarly, it can be interesting to know what happens to a functional when its input is changed by a small quantity.
The functional derivative (also known as a variational derivative) can be defined as follows.
\begin{definition}[Functional Derivative]
  \label{definition:functional_derivative}
  Let $J$ be a functional of $f$ which is in turn a function of $x \in \mathbb{R}$. Then the functional derivative is
  \begin{equation}
    \label{eq:definition_of_functional_derivative}
    \fdv{J[f(x)]}{f(x')} = \lim_{\epsilon\rightarrow 0} \frac{J[f(x) + \epsilon \delta(x - x')] - J[f(x)]}{\epsilon}
  \end{equation}
  if the limit exists. Here, $\delta(x)$ is the Dirac Delta function.
\end{definition}

For example, consider the functional derivative of \cref{eq:example_functional_definite_integral}. From \cref{definition:functional_derivative}
we have
\begin{equation*}
  \fdv{I}{f(x)} = \lim_{\epsilon\rightarrow 0} \frac{1}{\epsilon} \left(\int_{-\infty}^{\infty} f(x) + \epsilon \delta\left(x-x'\right)\,\mathrm{d}x 
      - \int_{-\infty}^{\infty} f(x)\,\mathrm{d}x\right),
\end{equation*}
which can be simplified to give
\begin{equation*}
  \fdv{I}{f(x)} = \lim_{\epsilon\rightarrow 0}\frac{1}{\epsilon}\int_{-\infty}^{\infty} \epsilon \delta(x-x')\,\mathrm{d}x = 1.
\end{equation*}

\section{Useful Results}
Two useful results related to the chain rule will now be proven.
\begin{theorem}
  \label{theorem:chain_rules}
  Let $f,g\colon\mathbb{R}\rightarrow\mathbb{R}$ be suitable functions. Then
  \begin{equation}
    \label{eq:chain_rule_functionals}
    \fdv{f(x)} \int_a^b g\left(f\left(y\right)\right)\,\mathrm{d}y = g'\left(f\left(x\right)\right)
  \end{equation}
  and
  \begin{equation}
    \label{eq:chain_rule_derivatives_functional}
    \fdv{f(x)} \int_a^b g\left(f'\left(y\right)\right)\,\mathrm{d}y = -\dv{x}\left(\dv{g\left(f'\left(x\right)\right)}{f'}\right)
  \end{equation}
  where conventional derivatives have been denoted with primes.
\end{theorem}
\begin{proof}
  Define
  \begin{equation*}
    J[f] = \int_a^b g\left(f\left(y\right)\right)\,\mathrm{d}y
  \end{equation*}
  Beginning with the first claim using \cref{definition:functional_derivative}:
  \begin{align*}
    \fdv{J}{f(x)}  &= \lim_{\epsilon\rightarrow 0} \frac{1}{\epsilon} \left(\int_a^b g\left(f(y) + \epsilon \delta\left(y-x\right)\right)\,\mathrm{d}y
                                                                     - \int_a^b g\left(f\left(y\right)\right)\mathrm{d}y \right)\\
    &=  \lim_{\epsilon\rightarrow 0} \frac{1}{\epsilon} \left(\int_a^b g\left(f(y)\right) + \epsilon \delta\left(y-x\right)g'\left(f\left(y\right)\right)\,\mathrm{d}y
                                                                     - \int_a^b g\left(f\left(y\right)\right)\,\mathrm{d}y\right)
  \end{align*}
  where the function $g$ has been Taylor expanded to first order. The second order terms will vanish in the limit and can therefore be ignored.
  Taking the limit gives
  \begin{align*}
    \fdv{f(x)} \int_a^b g\left(f\left(y\right)\right)\,\mathrm{d}y &= \int_a^b \delta(y - x) g'(f(y))\,\mathrm{d}y \\
                                                                   &= g'(f(x)).                                                              
  \end{align*}
  Defining
  \begin{equation*}
    J[f] =  \int_a^b g\left(f'\left(y\right)\right)\,\mathrm{d}y,
  \end{equation*}
  the proof of \cref{eq:chain_rule_derivatives_functional} proceed similarly for the second claim:
  \begin{align*}
    \fdv{J}{f(x)}  &= \lim_{\epsilon\rightarrow 0} \frac{1}{\epsilon}\left(\int_a^b g\left(\pdv{y}\left(f(y) + \epsilon \delta (y-x)\right)\right)\, \mathrm{d}y- \int_a^b g \left( f' \left( y \right) \right)\mathrm{d}y\right)       \\
                                                                    &= \lim_{\epsilon\rightarrow 0} \frac{1}{\epsilon}\left(\int_a^b g(f'(y)) + \epsilon \delta' (y-x)\dv{g(f'(y))}{f'}\, \mathrm{d}y- \int_a^b g(f'(y))\,\mathrm{d}y\right) \\
                                                                    &= \int_a^b \delta' (y-x)\dv{g(f'(y))}{f'}\, \mathrm{d}y \\
    &= - \dv{x} \dv{g(f'(x))}{f'}
  \end{align*}
  after integrating by parts.
\end{proof}

An important special case of \cref{theorem:chain_rules} is
\begin{equation}
  \label{eq:derivative_of_squared_derivative}
  \pdv[2]{f}{x} = -\fdv{f(x)}\int_a^b \frac{1}{2}\left(\pdv{f}{y}\right)^2\,\mathrm{d}y,
\end{equation}
which explains the sign in \cref{eq:spatial_system_energy}.

\section{Calculation of Correlation Function}
\label{sec:evaluation_of_integral}
Along with the notion of functional differentiation, there is also a notion of functional integration, in which a functional is integrated over
possible input functions. In this section,  In this section, I derive \cref{eq:two_point_correlation_actual} using some methods of functional integration.

The correlation function can be derived by taking functional derivatives of the functional integral given by
 \begin{equation}
   \label{eq:path_integral}
   Z = \int \exp\left( -\frac{2}{\sigma^2} \mathcal{H} \right) \, \mathcal{D}z,
 \end{equation}
 where
 \begin{equation}
   \label{eq:hamiltonian}
   \mathcal{H}[z] = \int_0^L \frac{1}{2}\lambda z(x)^2 + \frac{1}{2}D\left(\pdv{z}{x}\right)^2 - B(x)z(x)\dd{x}.
 \end{equation}

 To attack this integral, the tactic will be to shift to Fourier space, in which the part of the integral that survives differentiation can be factored out.
 In Fourier space, \cref{eq:hamiltonian} becomes
 \begin{equation}
   \label{eq:fourier_hamiltonian}
   \mathcal{H}[\{z_k\}] = \frac{1}{L^2}\sum_{n,m}\left(\frac{1}{2}\left(\lambda  - D nm\right) z_nz_m - B_mz_n\right)\int_0^L  e^{i(n+m)x}\dd{x}.
 \end{equation}
 The integral over $x$ is non-zero only when $m = -n$, so the sum over $m$ can be eliminated yielding
 \begin{equation}
   \label{eq:fourier_hamiltonian_contracted}
   \mathcal{H}[\{z_k\}] = \frac{1}{L}\sum_{n} \left(\frac{1}{2}\left(\lambda  + D n^2\right) z_nz_{-n} - B_{-n}z_{n}\right).
 \end{equation}
 Using the fact that $z \in \mathbb{R}$, which means that $z_n = z_{-n}^*$, \cref{eq:fourier_hamiltonian_contracted} can be simplified to
 \begin{equation}
   \label{eq:fourier_hamiltonian_contracted_mod}
   \mathcal{H}[\{z_k\}] = \frac{1}{L}\sum_{n} \left(\frac{1}{2}\left(\lambda  + D n^2\right) \left|z_n\right|^2- B_{-n}z_{n}\right).
 \end{equation}
 Setting $\phi_k = z_k - \frac{B_k}{Dk^2 + \lambda}$ and completeing the square means $\mathcal{H}$ is now
 \begin{equation}
   \label{eq:fourier_phi}
   \mathcal{H}[\{\phi_k\}] =  \frac{1}{L}\sum_{n} \left(\frac{1}{2}\left(\lambda + Dn^2\right) \left|\phi_n\right|^2 - \frac{1}{2}\frac{\left|B_n\right|^2}{\lambda + Dn^2}\right).
 \end{equation}
 This can be written as $\mathcal{H}[\{\phi_k\}] = \mathcal{H}_0[\{\phi_k\}] - \frac{1}{2L}\sum_{n}\frac{\left|B_n\right|^2}{\lambda + Dn^2}$, so that \cref{eq:path_integral} becomes
 \begin{equation}
   \label{eq:path_integral_factored}
   Z = \int \prod_{k>0} \exp \left(-\frac{2}{\sigma^2} \mathcal{H}_0[\{\phi_k\}]\right) \dd{\phi_k}\exp\left(\frac{1}{\sigma^2} \frac{1}{L}\sum_{n}\frac{\left|B_n\right|^2}{\lambda + Dn^2}\right)
 \end{equation}
 or
 \begin{equation}
   \label{eq:path_integral_factored_Z0}
   Z = Z_0 \exp\left(\frac{1}{\sigma^2} \frac{1}{L}\sum_{n}\frac{\left|B_n\right|^2}{\lambda + Dn^2}\right).
 \end{equation}
 Inverting $B_k$ back into real space gives
 \begin{equation}
   \label{eq:path_integral_factored_Z0_real}
   Z = Z_0 \exp\left(\frac{1}{\sigma^2} \iint B(x_1) g(x_1 - x_2) B(x_2) \dd{x_1}\dd{x_2}\right).
 \end{equation}
 where
 \begin{equation}
   \label{eq:G_from_Z}
   g(x_1 - x_2) =  \frac{1}{L}\sum_{n} \frac{e^{in(x_1-x_2)}}{\lambda + Dn^2}. 
 \end{equation}
 
 This sum can be evaluated by replacing it with an integral, assuming $L$ is large enough:
 \begin{equation}
   \label{eq:G_as_integral}
   g(x) = \frac{1}{2\pi}\int_{-\infty}^{\infty}  \frac{e^{ikx}}{\lambda + Dk^2} \dd{k} = \frac{1}{2} \frac{\xi}{D} e^{-x/\xi},
 \end{equation}
 hence using \cref{eq:two_point_from_partition},
 \begin{equation}
   \label{eq:G_from_G}
   G(x) = \frac{1}{2}\sigma^2g(x) =  \frac{1}{4} \sigma^2 \frac{\xi}{D} e^{-x/\xi}.
 \end{equation}
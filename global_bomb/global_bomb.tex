\chapter{The Compost Bomb and the Terrestrial Carbon Cycle}
\label{chapter:global_bomb}
\graphicspath{{global_bomb/figs/}}

\todo[inline]{This is basically just going to be equations for now}

In \cref{chapter:continuous_compost_bomb}, I introduced a vertical dimension into a model of the compost bomb. It was shown that this doesn't
surpress the compost bomb. That analysis was local ---  the modeling was done at a single point on the Earth's surface. It would be interesting to see
what a globally-averaged model that included biogeochemical heating would predict. Work by~\cite{Cox2006} suggested that the terrestrial carbon cycle is stable
only for certain parameter ranges. I will begin by investigating what effect biogeochemical heating has on this.

\section{Compost Bomb Bifurcation Analysis}

\subsection{Dynamical Equations}
The compsot bomb equations, introduced by~\cite{Luke2011}, are
\begin{subequations}
  \label{eq:compost_bomb_equations}
  \begin{align}
    c \dv{T_s}{t} &= - \kappa \left(T_s - T_a\right) + Ar_0C_se^{\alpha T_s} \label{eq:compost_bomb_soil_temperature} \\
    \dv{C_s}{t} &= \Pi - r_0C_se^{\alpha T_s}, \label{eq:compost_bomb_soil_carbon}
  \end{align}
\end{subequations}
where $T_s$ and $C_s$ are soil temperature and carbon, $T_a$ is the atmospheric temperature, $\alpha = 0.1\log Q_{10}$ is the sensitivity of
temperature to respiration, $\Pi$ is Net Primary Productivity, $A$ is the heat released by respiration, $\kappa$ is a heat transfer coefficent,
$c$ is a heat capacity and $r_0$ is the specific rate of respiration.

While the analysis in~\cite{Luke2011} was local, it will be assumed that \cref{eq:compost_bomb_equations} hold at the global scale too, where quanities are
replaced by their global value. It will also be assumed that Net Primary Productivity, $\Pi$, is an increasing function of atmospheric \ce{CO2}, as is $T_a$. 

There is a timescale separation in \cref{eq:compost_bomb_equations}. The dynamic of soil carbon are relatively slow, with a soil turn over time measured in the decades \parencite{Varney2022},
whereas the dynamics of soil temperature are much faster, reaching an equilibrium with the air temperature on the time scale of a day \parencite{Best2005}. With this in mind, the soil
temperature can be set to equilibrium by setting \cref{eq:compost_bomb_soil_temperature} equal to zero. This gives
\begin{equation*}
  0 = - \kappa \left(T_s - T_a\right) + Ar_0C_se^{\alpha T_s}
\end{equation*}
which can be solved for the soil temperature to  give
\begin{equation}
  \label{eq:soil_temperature_equilibrium}
  T_s = T_a - \frac{1}{\alpha} W\left(-\frac{Ar_0C_s \alpha e^{\alpha T_a}}{\kappa} \right),
\end{equation}
where $W(x)$ is the Lambert $W$ function \parencite{Corless1996}. The Lambert $W$ function, plotted in \cref{fig:lambert_W}, is defined as the solution to the equation
\begin{equation}
  \label{eq:lambert_W}
  W(x)e^{W(x)} = x.
\end{equation}

\begin{figure}
  \centering
  \begin{tikzpicture}
    \begin{axis}[
      %xmin=-1,
      %xmax=4,
      enlarge y limits=true,
      enlarge x limits=true,
      axis lines=middle,
      xlabel=$x$,
      xlabel style={xshift=0.5cm, yshift=-0.2cm},
      ylabel=$W(x)$,
      ylabel style={xshift=-0.5cm, yshift=0.6cm},
      samples=500]
      \addplot[domain=-4.2:-1,color=black] (x * exp(x), x);
      \addplot[domain=-1:1,color=black] (x * exp(x), x);
    \end{axis}
  \end{tikzpicture}
  \caption[The Lambert $W$ function]{The Lambert $W$ function, for $x \in [-1/e,e]$. The minimum value of $x$ for which $W(x)$ is real is $-1/e$. For $x < 0$, $W(x)$ is
  multivalued.}
  \label{fig:lambert_W}
\end{figure}

After noting that the argument of $W$ should be dimensionless, and that $r_0C_s$ has units of Net Primary Productivity, the quantity 
\begin{equation}
  \label{eq:critical_npp}
  \Pi_c = \frac{\kappa}{\alpha A}
\end{equation}
can be defined, which measures the influence of biogeochemical heating. No biogeochemical heating occurs for $A = 0$ or equivalently $\Pi_c = \infty$. Similarly
biogeochemical heating is strongest for $A \rightarrow \infty$ or again equivalently fpr $\Pi_c = 0$.

\Cref{eq:soil_temperature_equilibrium} can now be rewritten as
\begin{equation}
  \label{eq:soil_temperature_equilibrium_nppc}
  T_s = T_a - \frac{1}{\alpha} W\left(-\frac{r_0C_s e^{\alpha T_a}}{\Pi_c} \right).
\end{equation}
\Cref{eq:soil_temperature_equilibrium_nppc} can now be inserted into \cref{eq:compost_bomb_soil_carbon} to give
\begin{equation}
  \label{eq:soil_carbon_evolution}
  \dv{C_s}{t} = \Pi + \Pi_c W\left(-\frac{r_0C_s e^{\alpha T_a}}{\Pi_c} \right).
\end{equation}
which for given $\Pi$ and $T_a$ determines the evolution of $C_s$.

The implied equilibrium value for $C_s$ occurs when \cref{eq:soil_carbon_evolution} is zero which occurs when
\begin{equation}
  \label{eq:equilibrium_soil_carbon}
  C_s^{\mathrm{eq}} = \frac{\Pi}{r_0} e^{-\alpha T_a} e^{-\Pi/\Pi_c}.
\end{equation}
The no biogeochemcial heating case can be recovered by sending $\Pi_c \rightarrow \infty$ which gives $C_s^{\mathrm{eq}} = \frac{\Pi}{r_0}$.

\subsection{Closing the System}
It has been stated that $T_a$ and $\Pi$ are functions of atmospheric carbon, so to determine their behaviours requires specifying the behaviour of the rest of the carbon
cycle. The total carbon in the carbon cycle is conserved and so
\begin{equation}
  \label{eq:carbon_conservation}
  C_s + C_a + C_o = C_s^{\mathrm{eq}} + C_{a0} + C_{o0}
\end{equation}
where $C_a$ and $C_o$ are atmospheric and oceanic carbon and the quanitities on the righthand side are the equilibrium values.
Following~\cite{Cox2006}, it will be assumed that a fixed fraction, $\chi_0$, of atmospheric emissions reaches the ocean, meaning
\begin{equation}
  \label{eq:simple_ocean}
  C_a = C_{a0} -\frac{1}{1+\chi_0} (C_s - C_{s}^{\mathrm{eq}}).
  %C_s = C_{s}^{\mathrm{eq}} - (1 + \chi_0)(C_a - C_{a0}.
\end{equation}

Making the further assumption that atmospheric temperatures scale logarithmically with atmospheric \ce{CO2} \parencite{Pierrehumbert2010} gives
\begin{equation}
  \label{eq:atmospheric_temperatures}
  T_a = \frac{S}{\log 2} \log \frac{C_a}{C_{a0}} 
\end{equation}
where $S$ is the effective climate sensitivity experienced by the soils. Substituting \cref{eq:atmospheric_temperatures} into \cref{eq:soil_carbon_evolution} 
leads to
\begin{equation}
  \label{eq:global_soil_carbon}
  \dv{C_s}{t} = \Pi(C_a) + \Pi_c W\left(-\frac{r_0C_s}{\Pi_c} \left(\frac{C_a}{C_{a0}}\right)^\mu \right),
\end{equation}
where
\begin{equation}
  \label{eq:mu}
  \mu = \frac{\alpha S}{\log 2}
\end{equation}
and $C_a$ is determined through \cref{eq:simple_ocean}. By assumption, $C_s = C_s^{\mathrm{eq}}$ when $T_a = 0$ which means that 
$r_0 = \frac{\Pi_0}{C_s^{\mathrm{eq}}}e^{-\Pi_0/\Pi_c}$, where $\Pi_0 = \Pi\left(C_{a0}\right)$.

To numerically compute the equilibria of \cref{eq:global_soil_carbon}, the dependence of $\Pi$ on
$C_a$  must be set. It is chosen to be
\begin{equation}
  \label{eq:npp_fertilization}
  \Pi(C_a) = \frac{\Pi_{\infty} C_a}{C_a + C_{a_{1/2}}}.
\end{equation}
This is an increasing function of $C_a$, which saturates to $\Pi_{\infty}$ for $C_a \gg C_{a_{1/2}}$.

It is now straightforward to compute the bifurcation diagram, which is plotted in \cref{fig:compost_bomb_bif}. The figure shows the equilibrium soil carbon as a function
of $\mu$ for different values of $\Pi_c$. It can be seen that there is a transcritical bifurcation at a certain value of $\mu$. This value of $\mu$ decreases with decreasing $\Pi_c$.
\begin{figure}
  \centering
  \includegraphics[width=\textwidth,keepaspectratio]{compost_bomb_global_bifurcation}
  \caption[Global Compost Bomb Bifurcation Diagram]{The equilibrium state of \cref{eq:global_soil_carbon} for two values of $\Pi_c$ as a function of $\mu$.
    The two values of $\Pi_c$ represent a no biogeochemical heating case ($\Pi_c = \infty$) and a biogeochemical heating case ($\Pi_c = \SI{308}{\peta\gram\carbon\per\year}$).
    The other parameters were set to $\chi_0 = 0.25$, $\Pi_0 =\SI{55}{\peta\gram\carbon\per\year}$, $C_{1/2} = \SI{593.6}{\peta\gram\carbon}$, $C_{a0} = \SI{589}{\peta\gram\carbon}$,
    $C_{s}^{\mathrm{eq}} = \SI{1500}{\peta\gram\carbon}$.}
  \label{fig:compost_bomb_bif}
\end{figure}

\subsection{Computation of Bifurcation Point}
By construction, there is always an equilibrium at $C_s = C_s^{\mathrm{eq}}$. Its stability is determined by the jacobian of \cref{eq:global_soil_carbon}, which in this case amounts to
taking the derivative of the righthand side of \cref{eq:global_soil_carbon} when $C_s = C_s^{\mathrm{eq}}$. The system will transition from being stable to unstable when
\begin{equation*}
  \dv{\dot{C_s}}{C_s} = 0.
\end{equation*}
In order to calculate the derivative the identity
\begin{equation}
  \label{eq:derivative_of_lambert_W}
  W'(x) = \frac{W(x)}{x\left(1 + W\left(x\right)\right)}
\end{equation}
will be useful, which follows from taking the derivative of \cref{eq:lambert_W}.

If the derivative of the right hand side of \cref{eq:global_soil_carbon} is taken and set to zero, 
\begin{equation}
  \label{eq:soil_carbon_lambda}
  \dv{\Pi}{C_a}\dv{C_a}{C_s} + \frac{\Pi_c}{C_s^{\mathrm{eq}}} \frac{W\left(-\frac{r_0C_s^{\mathrm{eq}}}{\Pi_c}\right)}{1+W\left(-\frac{r_0C_s^{\mathrm{eq}}}{\Pi_c}\right)} \left(
    1 + \mu \frac{C_s^{\mathrm{eq}}}{C_{a0}} \dv{C_a}{C_s}\right) = 0
\end{equation}
is obtained. This can be rearranged for $\mu$ to give
\begin{equation}
  \label{eq:critical_mu}
  \mu = -\frac{C_{a0}}{C_{s}^{\mathrm{eq}}}\dv{C_s}{C_a}\left(\dv{\Pi}{C_a}\dv{C_a}{C_s} \frac{C_s^{\mathrm{eq}}}{\Pi_c}\frac{1+W\left(-\frac{r_0C_s^{\mathrm{eq}}}{\Pi_c}\right)}{W\left(-\frac{r_0C_s^{\mathrm{eq}}}{\Pi_c}\right)} + 1 \right)
\end{equation}
which simplifies to
\begin{equation}
  \label{eq:critical_mu_simple_ocean}
  \mu = \left(1+\chi_0\right) \frac{C_{a0}}{C_{s}^{\mathrm{eq}}} -
  \frac{C_{a0}}{\Pi_c}\frac{1+W\left(-\frac{r_0C_s^{\mathrm{eq}}}{\Pi_c}\right)}{W\left(-\frac{r_0C_s^{\mathrm{eq}}}{\Pi_c}\right)}\dv{\Pi}{C_a}.
\end{equation}
This can be further reduced, by using \cref{eq:equilibrium_soil_carbon}, to
\begin{equation*}
  \mu = \left(1+\chi_0\right) \frac{C_{a0}}{C_{s}^{\mathrm{eq}}} -
  \frac{C_{a0}}{\Pi_c}
  \frac{1+W\left(-\frac{\Pi_0}{\Pi_c}\exp\left(-\frac{\Pi_0}{\Pi_c}\right)\right)}
  {W\left(-\frac{\Pi_0}{\Pi_c}\exp\left(-\frac{\Pi_0}{\Pi_c}\right)\right)}\dv{\Pi}{C_a} 
\end{equation*}
and then by using the definition of the $W$ function, \cref{eq:lambert_W}, to give
\begin{equation*}
  \mu = \left(1+\chi_0\right) \frac{C_{a0}}{C_{s}^{\mathrm{eq}}} -
  \frac{C_{a0}}{\Pi_c}
  \frac{1-\frac{\Pi_0}{\Pi_c}}{-\frac{\Pi_0}{\Pi_c}}\dv{\Pi}{C_a}.
\end{equation*}
This can then be cleaned up to give a final form for the critical $\mu$ value, notated at $\mu^*$, which seperates stable from unstable soil
carbon states. $\mu^*$ is therefore
\begin{equation}
  \label{eq:critical_mu_simple_ocean_in_terms_of_npp}
  \mu^* = \left(1+\chi_0\right) \frac{C_{a0}}{C_{s}^{\mathrm{eq}}} +
  \frac{C_{a0}}{\Pi_0} \dv{\Pi}{C_a} - \frac{C_{a0}}{\Pi_c}\dv{\Pi}{C_a}.
\end{equation}
It is instructive to take the $\Pi_c \rightarrow \infty$ limit, which gives the behaviour in the no biogeochemical heating case:
\begin{equation}
  \label{eq:mu_infinity}
  \mu^*_{\infty} =\left(1+\chi_0\right) \frac{C_{a0}}{C_{s}^{\mathrm{eq}}} +
  \frac{C_{a0}}{\Pi_0} \dv{\Pi}{C_a}.
\end{equation}
Therefore the effect of biogeochemical heating is the reduce the critical $\mu$ value for which an instability occurs by the amount
\begin{equation}
  \label{eq:reduction_in_stability_due_to_biogeochemical}
  \frac{C_{a0}}{\Pi_c}\dv{\Pi}{C_a}.
\end{equation}
When $\Pi_c \rightarrow 0$ this reduction becomes infinite and no stable soil carbon state is possible.

\Cref{eq:critical_mu_simple_ocean_in_terms_of_npp} can be plotted, as has been done in \cref{fig:critical_mu_vs_pic}, to divide the $(\mu,\Pi_c)$ parameter plane
into stable and unstable regions, for a range of $\dv*{\Pi}{C_a}$ values. For a $33\%$ increase in NPP due to doubling \ce{CO2}
\parencite{Wenzel2016}, then $\dv*{\Pi}{C_a}\approx \SI{0.05}{\per\year}$.

\begin{figure}
  \centering
  \includegraphics[width=\textwidth,keepaspectratio]{bifurcation_parameter_plane}
  \caption[The ($\mu$,$\Pi_c)$ parameter plane]{A plot of the $(\mu,\Pi_c)$ parameter plane, where the lines are drawn according to \cref{eq:critical_mu_simple_ocean_in_terms_of_npp}
    for a range of different $\dv*{\Pi}{C_a}$ values. The lines separate the stable and unstable regions. The other parameters are $\chi_0 = 0.25$,$\Pi_0 =\SI{55}{\peta\gram\carbon\per\year}$,
    $C_{a0} = \SI{589}{\peta\gram\carbon}$ and $C_{s}^{\mathrm{eq}} = \SI{1500}{\peta\gram\carbon}$.}
  \label{fig:critical_mu_vs_pic}
\end{figure}

\section{Determining the Parameters}
\subsection{The Effective Climate Sensitivity, $S$}
A first guess to the value of $S$ may be the Equilibrium Climate Sensitivity, ECS, namely the increase in global mean surface temperatures when atmospheric \ce{CO2} concentrations
are doubled. However, the pattern of warming is spatially heterogeneous, with more warming happening over land than over oceans \parencite{Morice2021}. As soil carbon is found on land not
in the oceans, it may be better to view $S$ as the climate sensitivity over land.

In order to determine what $S$ should be, the soil carbon balance will be considered at every point on the Earth's surface. There will be a spatially varying amount of warming which will
lead to a change in global soil carbon. This will then be compared to what level of spatially uniform warming would be required to give the same change in soil carbon. By setting the spatially varying
pattern of warming to the pattern of warming caused by doubling \ce{CO2}, the resulting effective warming will be $S$.

Ignoring the effects of biogeochemical heating, the spatially resolved soil carbon balance can be written as:
\begin{equation}
  \label{eq:spatially_resolved_soil_carbon}
  \pdv{C_s(\bm{r},t)}{t} = \Pi(\bm{r},t) - r_0(\bm{r},t)C_s(\bm{r},t)e^{\alpha T_a(\bm{r},t)},
\end{equation}
where $\bm{r}$ is the position on the Earth's surface and $\Pi$, $r_0$ and $C_s$ are allowed to vary across. It is assumed that $\alpha$ is constant and that there is no
horizontal transport of soil carbon.

\Cref{eq:spatially_resolved_soil_carbon} can be averaged over space. Denoting spatial averages with $\langle \bullet \rangle$, this gives
\begin{equation}
  \label{eq:spatially_averaged_soil_carbon}
  \dv{\left\langle C_s\right\rangle}{t} = \left\langle \Pi \right \rangle - \left\langle r_0 C_s e^{\alpha T_a} \right \rangle.
\end{equation}
Assuming the warming is small, the exponential can be expanded to first order using $e^x \approx 1 + x$ leading to
\begin{align*}
  \dv{\left\langle C_s\right\rangle}{t} &\approx \left\langle \Pi \right \rangle - \left\langle r_0 C_s + r_0 C_s \alpha T_a \right \rangle \\
                                        &\approx \left\langle \Pi \right \rangle - \left\langle r_0 C_s\right \rangle - \alpha \left \langle r_0C_s T_a\right\rangle.
\end{align*}
Globally averaged respiration must balance globally averaged NPP in equilibrium, so $\langle \Pi \rangle = \langle r_0 C_s \rangle$. This means that
\begin{equation}
  \label{eq:soil_carbon_linearised}
  \dv{\langle C_s\rangle}{t} \approx -\alpha \langle \Pi T_a \rangle.
\end{equation}
Introducting an effective temperature $T_{\mathrm{eff}}$, defined so that
\begin{equation}
  \label{eq:motivation_of_effective_temperature}
  - \alpha \left \langle \Pi T_a \right\rangle = - \alpha \left \langle \Pi \right\rangle T_{\mathrm{eff}}
\end{equation}
which implies
\begin{equation}
  \label{eq:definition_of_effective_temperature}
  T_{\mathrm{eff}} = \frac{\left \langle \Pi T_s \right\rangle}{\left \langle \Pi \right\rangle}.
\end{equation}
After a doubling of \ce{CO2}, $T_{\mathrm{eff}} = S$ and $\langle T_a \rangle = \mathrm{ECS}$ which means that
\begin{equation}
  \label{eq:S_vs_ECS}
  \frac{S}{\mathrm{ECS}} = \frac{\left \langle \Pi T_a \right\rangle}{\left \langle \Pi \right\rangle \left \langle T_a \right \rangle}.
\end{equation}
In words, this means $S$ is given by an NPP weighted average of global temperatures. A simple estimate of this ratio can be made by noting that $\Pi$ is zero over ocean.
Assuming that over land the correlation between NPP and $T_a$ is weak then
\begin{equation}
  \label{eq:S_vs_ECS_land_ocean}
  \frac{S}{\mathrm{ECS}}
  = \frac{\left\langle \Pi\right\rangle_{\mathrm{land}} \left\langle T_a \right\rangle_{\mathrm{land}}}{\left \langle \Pi \right\rangle_{\mathrm{land}} \left \langle T_a \right \rangle_{\mathrm{global}}}
  = \frac{\left\langle T\right\rangle_{\mathrm{land}}}{\left\langle T \right\rangle_{\mathrm{global}}}
\end{equation}
which means that $S$ is the climate sensitivity over land, as predicted by the rought guess. The IPCC find that there has been about $1.5$ times more warming over land than over the globe
\parencite{AR6}.

\Cref{eq:S_vs_ECS} can be used with abrupt-4xCO2 CMIP runs \parencite{Eyring2016} to estimate $S$. Taking $\Pi$ to be the inital NPP in these simulations gives an estimate of $S$ of around $1.5$,
as will be shown in \cref{tab:S_vs_ECS}.

\subsection{The influence of biogeochemical heating, $\Pi_c$}
\todo[inline]{Do with Physically based --- with Penman Monteith - this is giving me a constraint on the ratio?}
The quantity $\Pi_c$, which measures the role of biogeochemical heating on the global carbon cycle depends on the sensitivity of heterotrophic respiration to
temperature, the heat released by respiration and the conducivity between the land and the atmosphere.

The first of these, $\alpha$, is reasonably well known. In terms of $Q_{10}$ it is $\alpha = 0.1 \log Q_{10}$. $Q_{10}$ is usually taken to be about $2$
\parencite{Jones2001,Clark2011}. This means that $\alpha \approx 0.07$.

The second quantity, $A$, can be estimated biochemically. Its value is taken to be $A = \SI{3.9E7}{\joule\per\kilo\gram\carbon}$ \parencite{Luke2011}.

The final quanity to estimate is the conductivity between the land and the atmosphere. This varies from region to region, depending on the land surface cover
\parencite{Beringer2001}, the soil hydrology \parencite{Dharssi2009} and the soil type \parencite{Best2011}. As such an effective conductivity will be required.
% It can be obtained using ERA5 data \parencite{Hersbach2020}. To do this, first take the  energy balance at the surface \parencite{Hartmann2015}, which can be written as
% \begin{equation}
%   \label{eq:surface_energy_balance}
%   G = R - H - LE
% \end{equation}
% where $G$ is the soil heat flux, $R$ the net radiation, and $H$ and $LE$ are the sensible and latent heat fluxes. Then as the skin temperature $T_*$ is related to the soil
% temperature \parencite{Best2005} of the top layer, $T_1$, by
% \begin{equation}
%   \label{eq:skin_temp}
%   T_* = T_1 + \frac{1}{\kappa} G
% \end{equation}
% an estimate of $\kappa$ can be made using the variables $R,L,LE,T_*$ and $T_1$ which are all avaliable from ERA5. Taking annual averages over land of these variables for the period 2000 to 2020
% gives a value of $\kappa = \SI{85.0}{\watt\per\square\meter\per\kelvin}$, which implies a value of $\Pi_c \approx \SI{15000}{\peta\gram\carbon\per\year}$. This value is much larger than typical
% values of NPP and so this biogeochemical heating effect must be small at the global scale. However, it should be noted that $\kappa$ can be smaller in, for example, well insulated mossy soils
% and as such the biogeochemical heating may still be important regionally.
The heat transfer coeffeicent can be estimated from the conductivity via
\begin{equation}
  \label{eq:conductivity_via_heat_transfer}
  \kappa = \frac{2\lambda}{\Delta z}
\end{equation}
where $\Delta z$ is a soil thickness and $\lambda$ is the conductivity. Taking $\Delta z = \SI{0.1}{\meter}$ and $\lambda = \SI{0.227}{\watt\per\meter\per\kelvin}$
\parencite{Cox1999} gives a value of $\kappa = \SI{4.5}{\watt\per\meter\squared\per\kelvin}$ and a value of $\Pi_c = \SI{8000}{\peta\gram\carbon\per\year}$. This value is much larger than typical
values of NPP and so this biogeochemical heating effect must be small at the global scale. However, it should be noted that $\kappa$ can be smaller in, for example, well insulated mossy soils
and as such the biogeochemical heating may still be important regionally.


\section{Rate Dependence and Transient Behaviour}
\begin{subequations}
  \label{eq:compost_bomb_equations_scaled}
  \begin{align}
    \tau\dv{\theta}{t} &= -\left(\theta - \mu \log \frac{C_a}{C_{a0}}\right) + \frac{C_s}{C_s^{\mathrm{eq}}} \frac{\Pi_0}{\Pi_c} e^{-\frac{\Pi_0}{\Pi_c}} e^{\theta} \\
    \dv{C_s}{t}      &= \Pi(C_a) - \frac{C_s}{C_s^{\mathrm{eq}}} \Pi_0 e^{-\frac{\Pi_0}{\Pi_c}} e^{\theta} \\
    C_a           &= C_{a0} - \frac{C_s - C_s^{\mathrm{eq}}}{1+\chi_0} + \frac{rt}{1+\chi_0}
  \end{align}
\end{subequations}
where $\theta = \alpha T_s$ and $\tau = c/\kappa$.

\subsection{The Effect of biogeochemical heating on atmospheric \ce{CO2}}
To investigate the contribution of biogeochemical heating to global warming, it is necessary to determine how much carbon reaches the atmosphere compared to the case where there is no
biogeochemical heating. In order to do this, \cref{eq:compost_bomb_equations_scaled} were integrated for 100 years with $r = \SI{10.0}{\peta\gram\carbon\per\year}$. This roughly
corresponds to recent emission rates \parencite{Friedlingstein2022}.

These integratations where performed for a range of parameter values. The range of $\mu$ was \SIrange{0.1}{1.2}{}, the range of $C_{1/2}$ was \SIrange{0.0}{1000.0}{\ppm} and
the range of $\Pi_c$ was \SIrange{100}{10000}{\peta\gram\carbon\per\year}. The results were then compared to simulations with $\Pi_c = \infty$, the no biogeochemical heating case and the difference
in atmospheric \ce{CO2} between the two cases were calculated.
\begin{figure}
  \centering
  \includegraphics[width=\textwidth,keepaspectratio]{extra_co2_due_to_bgc}
  \caption[The effect of biogeochemical heating on atmospheric \ce{CO2}]{The difference, in $\SI{}{\peta\gram\carbon\per\year}$, between atmospheric \ce{CO2} in a simulation
    with biogeochemical heating and without biogeochemical heating. The hatched regions show the unstable region of parameter space as determined
    by \cref{eq:critical_mu_simple_ocean_in_terms_of_npp}. The value of $C_{1/2}$ here is \SI{280}{\ppm}.}
  \label{fig:biogeochem_atmos}
\end{figure}

The case when $C_{1/2} = \SI{280}{\ppm}$ is shown as a contour plot in \cref{fig:biogeochem_atmos}. It shows that as the role of biogeochemical heating is increased (corresponding to decreasing
$\Pi_c$) more \ce{CO2} accumulates in the atmosphere. It also shows that as $\mu$ is increased, corresponding to increasing climate sensitivity or increasing $Q_{10}$, initially the amount of extra
\ce{CO2} increases, but beyond a certain level of $\mu$ this amount decreases but remains positive.

This can be explained as follows. For $\mu = 0$, ignoring \ce{CO2} fertilisation effects,
both of the biogeochemical and non-biogeochemical cases will experience no changes in soil carbon and so should see little differences in the amount of $\ce{CO2}$ in the atmosphere. 
When is very $\mu$ is large, both cases can lose all the soil carbon leading to the same overall change in atmospheric carbon. It is therefore for intermediate $\mu$ values that biogeochemical
heating makes the biggest difference to climate change.

\begin{figure}
  \centering
  \subfloat[\centering $C_{1/2} = \SI{0}{\ppm}$]{{\includegraphics[width=0.5\textwidth,keepaspectratio]{extra_co2_due_to_bgc_no_fert}}\label{fig:no_co2_fert}}
  \subfloat[\centering $C_{1/2} = \SI{1000}{\ppm}$]{{\includegraphics[width=0.5\textwidth,keepaspectratio]{extra_co2_due_to_bgc_with_fert}}\label{fig:with_co2_fert}}
  \caption[The role of \ce{CO2} fertilsation in the effect of biogeochemical heating on atmospheric \ce{CO2}]{
    \Cref{fig:no_co2_fert} shows a similar parameter plane to \cref{fig:biogeochem_atmos}, except the \ce{CO2} fertilisation effect has been turned off. \Cref{fig:with_co2_fert} shows
    the case with a strong \ce{CO2} fertilisation effect. The hatched region again shows regions of the parameter plane where there is no stable state.}
  \label{fig:biogeochem_co2_fert}
\end{figure}

\Cref{fig:biogeochem_co2_fert} helps illustrate the role the \ce{CO2} fertilisation effect plays in this extra amount of \ce{CO2} that reaches the atmosphere due to biogeochemical heating.
Both panels show a similar qualitative behaviour to \cref{fig:biogeochem_atmos}. However the magnitude is different. In particular the stronger the \ce{CO2} fertilisation effect, the bigger a
difference biogeochemical heating makes to climate change. This is because a stronger \ce{CO2} fertilisation effect leads to a larger input of carbon to the soil and thus a larger amount
of soil respiration which through biogeochemical heating leads to even more respiration and thus a larger amount of atmospheric \ce{CO2}.

For a $\mu$ vale of around $0.5$, and $\Pi_c = \SI{8000}{\peta\gram\carbon\per\year}$, then biogeochemical heating contributes somewhere between $\SIrange{3}{4}{\peta\gram\carbon}$
to climate change over this 100 year period, assuming $C_{1/2} = \SI{280}{\ppm}$. In other words, if humanity makes no changes to its emission rates, then biogeochemical alone
would contribute as much to global warming in the next hundred years as humanity did in one year in the late 1960s \parencite{Friedlingstein2022}.

\section{Conclusions}
\chapter{The Compost Bomb and the Terrestrial Carbon Cycle}
\graphicspath{{global_bomb/figs/}}

\todo[inline]{This is basically just going to be equations for now}

Following\cite{Luke2011}, we have the compost comb equations
\begin{subequations}
  \label{eq:compost_bomb_equations}
  \begin{align}
    \mu \dv{T_s}{t} &= - \kappa \left(T_s - T_a\right) + Ar_0C_se^{\alpha T_s} \label{eq:compost_bomb_soil_temperature} \\
    \dv{C_s}{t} &= \Pi - r_0C_se^{\alpha T_s} \label{eq:compost_bomb_soil_carbon}
  \end{align}
\end{subequations}

Noting the obvious timescale seperation in \cref{eq:compost_bomb_equations} we can put \cref{eq:compost_bomb_soil_temperature} into
equilibrium to get
\begin{equation*}
  0 = - \kappa \left(T_s - T_a\right) + Ar_0C_se^{\alpha T_s}
\end{equation*}
which can be solved for the soil temperature to  give
\begin{equation}
  \label{eq:soil_temperature_equilibrium}
  T_s = T_a - \frac{1}{\alpha} W\left(-\frac{Ar_0C_s \alpha e^{\alpha T_a}}{\kappa} \right),
\end{equation}
where $W(x)$ is the Lambert $W$ function. The Lambert $W$ function is defined as the solution to the equation
\begin{equation}
  \label{eq:lambert_W}
  W(x)e^{W(x)} = x.
\end{equation}

\begin{figure}
  \centering
  \begin{tikzpicture}
    \begin{axis}[
      xmin=-1,
      xmax=4,
      enlarge y limits=false,
      axis lines=left,
      xlabel=$x$,
      ylabel=$W(x)$,
      samples=50]
      \addplot[domain=-5:-1] (x * exp(x), x);
      \addplot[domain=-1:2] (x * exp(x), x);
    \end{axis}
  \end{tikzpicture}
  \caption{The Lambert $W$ function}
  \label{fig:lambert_W}
\end{figure}
The Lambert $W$ is plotted in \cref{fig:lambert_W}.

Defining
\begin{equation}
  \label{eq:critical_npp}
  \Pi_c = \frac{\kappa}{\alpha A}
\end{equation}
we can rewrite \cref{eq:soil_temperature_equilibrium} as
\begin{equation}
  \label{eq:soil_temperature_equilibrium_nppc}
  T_s = T_a - \frac{1}{\alpha} W\left(-\frac{r_0C_s e^{\alpha T_a}}{\Pi_c} \right).
\end{equation}
We should therefore insert \cref{eq:soil_temperature_equilibrium_nppc} in \cref{eq:compost_bomb_soil_carbon}
to give
\begin{equation}
  \label{eq:soil_carbon_evolution}
  \dv{C_s}{t} = \Pi + \Pi_c W\left(-\frac{r_0C_s e^{\alpha T_a}}{\Pi_c} \right).
\end{equation}
This implies the equilibrium value of $C_s$ is
\begin{equation}
  \label{eq:equilibirum_soil_carbon}
  C_s^{\mathrm{eq}} = \frac{\Pi}{r_0} e^{-\alpha T_a} e^{-\Pi/\Pi_c}.
\end{equation}
Note that we recover the no biogeochemcial heating case by sending $\Pi_c \rightarrow \infty$.


Recognising that
\begin{equation}
  \label{eq:atmospheric_temperatures}
  T_a = \frac{S}{\log 2} \log \frac{C_a}{C_{a0}} 
\end{equation}
leads, upon substituting \cref{eq:atmospheric_temperatures} into \cref{eq:soil_carbon_evolution},
\begin{equation}
  \label{eq:global_soil_carbon}
  \dv{C_s}{t} = \Pi + \Pi_c W\left(-\frac{r_0C_s}{\Pi_c} \left(\frac{C_a}{C_{a0}}\right)^\mu \right),
\end{equation}
where
\begin{equation}
  \label{eq:mu}
  \mu = \frac{\alpha S}{\log 2}.
\end{equation}
We will assume
\begin{equation}
  \label{eq:npp_fertilization}
  \Pi(C_a) = \Pi_{\infty}\frac{C_a}{C_a + C_{a_{1/2}}}.
\end{equation}
We will assume further that a fixed fraction $\chi_0$ of atmospheric emissions reaches the ocean, meaning
\begin{equation}
  \label{eq:simple_ocean}
  C_a = C_{a0} -\frac{1}{1+\chi_0} (C_s - C_{s}^{\mathrm{eq}})
  %C_s = C_{s}^{\mathrm{eq}} - (1 + \chi_0)(C_a - C_{a0}.
\end{equation}
We chose the temperature anomaly so that $T_a = 0$ corresponds to an equilibrium. We can then set
$r_0 = \frac{\Pi}{C_s^{\mathrm{eq}}}e^{-\Pi/\Pi_c}$ giving
\begin{equation}
  \label{eq:soil_carbon_evolution_with_r0_fixed}
  \dv{C_s}{t} = \Pi + \Pi_c W\left(-\frac{C_s}{C_s^{\mathrm{eq}}}\frac{\Pi}{\Pi_c}e^{-\Pi/\Pi_c} \left(\frac{C_a}{C_{a0}}\right)^\mu \right).
\end{equation}
% Combining \cref{eq:soil_carbon_evolution_with_r0_fixed,eq:simple_ocean,eq:npp_fertilization} gives
% \begin{equation}
%   \label{eq:soil_carbon_evolution_combined}
%   \dv{C_s}{t} = \Pi_{\infty}\frac{C_{a0} -\frac{1}{1+\chi_0} (C_s - C_{s}^{\mathrm{eq}})}{C_{a0} -\frac{1}{1+\chi_0} (C_s - C_{s}^{\mathrm{eq}})+ C_{a_{1/2}}}
%   + \Pi_c W\left(-\frac{C_s}{C_s^{\mathrm{eq}}}\frac{\Pi}{\Pi_c}e^{-\Pi/\Pi_c} \left(\frac{C_{a0} -\frac{1}{1+\chi_0} (C_s - C_{s}^{\mathrm{eq}})}{C_{a0}}\right)^\mu \right).
% \end{equation}
To find the bifurcation we therefore just have to calculate where
\begin{equation*}
  \dv{\dot{C_s}}{t} = 0,
\end{equation*}
we will make use of
\begin{equation}
  \label{eq:derivative_of_lambert_W}
  W'(x) = \frac{W(x)}{x\left(1 + W\left(x\right)\right)}.
\end{equation}
This leads to
\begin{equation}
  \label{eq:critical_mu}
  \mu^* = \left(1 + \chi_0\right) \frac{C_{a0}}{C_s^{\mathrm{eq}}} \left( \frac{1}{1+\chi_0}\left( \frac{1}{W\left(-\frac{\Pi}{\Pi_c}e^{-\Pi/\Pi_c}\right)}
      - \frac{\Pi_c}{\Pi}\right)\frac{C_s^{\mathrm{eq}}}{\Pi_c}\dv{\Pi}{C_a} - 1 \right)
\end{equation}
where $\mu^*$ is the value of $\mu$ where the bifurcation takes place.

% We have
% \begin{equation}
%   \label{eq:derivative_of_cs}
%   \dv{\dot{C_s}}{C_s} = \dv{\Pi}{C_s} + \frac{\Pi_c W\left(\frac{\Pi}{\Pi_c}e^{\Pi/\Pi_c}\right)}{1+W\left(\frac{\Pi}{\Pi_c}e^{\Pi/\Pi_c}\right)}\left(\frac{1}{\Pi\Pi_c}\dv{\Pi}{C_s} - \frac{1}{C_s^{\mathrm{eq}}} - \frac{\mu}{C_{a0}}\right) = 0.
% \end{equation}
% As
% \begin{equation}
%   \label{eq:derivative_of_npp}
%   \dv{\Pi}{C_s} = -\frac{\Pi_{\infty}C_{a_{1/2}}}{\left(C_{a0} + C_{a_{1/2}}\right)^2}
% \end{equation}
% we have
% \begin{equation}
%   \label{eq:derivative_of_cs_npp}
%   -\frac{\Pi_{\infty}C_{a_{1/2}}}{\left(C_{a0} + C_{a_{1/2}}\right)^2} - \frac{\Pi_c W\left(\frac{\Pi}{\Pi_c}e^{\Pi/\Pi_c}\right)}{1+W\left(\frac{\Pi}{\Pi_c}e^{\Pi/\Pi_c}\right)}\left(\frac{1}{\Pi\Pi_c}\frac{\Pi_{\infty}C_{a_{1/2}}}{\left(C_{a0} + C_{a_{1/2}}\right)^2} + \frac{1}{C_s^{\mathrm{eq}}} + \frac{\mu}{C_{a0}}\right) = 0.
% \end{equation}
% So there is a bifurcation when
% \begin{equation}
%   \label{eq:bifurcation_mu}
%   \mu = -\frac{\Pi_{\infty}C_{a_{1/2}}C_{a0}}{\left(C_{a0} + C_{a_{1/2}}\right)^2}\frac{1+W\left(\frac{\Pi}{\Pi_c}e^{\Pi/\Pi_c}\right)}{\Pi_c W\left(\frac{\Pi}{\Pi_c}e^{\Pi/\Pi_c}\right)}-
%   \frac{1}{\Pi\Pi_c}\frac{\Pi_{\infty}C_{a_{1/2}}C_{a0}}{\left(C_{a0} + C_{a_{1/2}}\right)^2} - \frac{C_{a0}}{C_s^{\mathrm{eq}}} 
% \end{equation}\todo{Is this correct? I don't think it has the right $\Pi_c$ goes to infinity limit}
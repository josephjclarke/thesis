\chapter{Preface}

I arrived in Exeter to start work on my PhD in September 2019, hoping to learn something about tipping points. I began by looking at the compost
bomb --- a somewhat esoteric rate-induced tipping point. Just six months later, the government declared a national lockdown
and I spent the next year or so working from home. I think this explains my research on early warning signals --- it's relatively easy to
do that from home. My research then, has two main strands: one that investigates the compost bomb and climate-carbon system instability more broadly
and another that looks at early warning signals.

\Cref{sec:intro1,sec:intro2} serve as an introduction to the thesis. In \cref{sec:intro1} I outline the climate-carbon system and discuss some potential
tipping points in the Earth system. In \cref{sec:intro2} I develop some of the theory about tipping points and early warning signals. Much of this is
quite general but I draw on examples from the Earth system.

\Cref{chapter:continuous_compost_bomb} is an investigation into the compost bomb. I did some of that work during the summer of 2020, when Siberia
was experiencing a spate of wildfires and stories of `zombie fires' circulated. This is why the potential for a hot summer to trigger compost bombs
was a focus of the chapter.

In \cref{chapter:global_bomb} I considered the compost bomb feedback at the global scale and its effect on the carbon cycle.
The fact that there was the potential for instability combined with the need to update some older research motivated a more thorough investigation of
the stability of the climate-carbon system. This was done in \cref{chapter:conceptual_carbon_cycle}.

\Cref{chapter:spatial_ews,chapter:rosa} contain the work on early warning signals. My desire to draw out the analogies between Earth system
tipping points and the established physics of phase transitions lead to work on spatial early warning signals, which can be found in \cref{chapter:spatial_ews}.
I became concerned that many applications of early warning signals were really detecting changes in what drives variability in the system, rather than the stability
of the system itself. I investigated this in \cref{chapter:rosa}.

In the final chapter, \cref{chapter:conclusions} I summarise my work and suggest some future research directions.

So far two of these chapters, \cref{chapter:continuous_compost_bomb,chapter:rosa}, have become published papers. They appear in this thesis mostly unchanged
aside from a few typographical alterations and the inclusion of supplementary material into the main text. The code used in these papers can be found online. 
For \cref{chapter:continuous_compost_bomb} it is located at \url{www.github.com/josephjclarke/ContinuumCompostBomb} and for
\cref{chapter:rosa} at \url{https://github.com/josephjclarke/BeyondWhiteNoiseEWS}.

For historical reasons Exeter climate scientists are mostly based in the department of mathematics. This has generally been an edifying experience and has influenced the
direction of my PhD in all sorts of ways. One of these ways is that I use $\log$ to mean the natural logarithm, unless otherwise stated. I am however a physicist
at heart the rest of my notational choices reflect that.

\begin{flushright}
  Joe Clarke\\
  Exeter\\
  August 2023
\end{flushright}